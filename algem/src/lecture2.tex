\lectureinfo{2}{11.02.2026}

\knt{
  $\ker \phi, \im \phi$ --- $\phi$-инвариантны.
}

\jprp{
  Пусть $\phi \in \CL(V)$. Тогда $\forall U \le \ker \phi$ ---
  $\phi$-инвариантно, и $\forall W \ge \im \phi$ --- $\phi$-инвариантен.
}

\jprf{\
  \jen{
    \item Если $U \le \ker \phi$, то $\phi(U) = 0 \le U$.
    \item Если $W \ge \im \phi$, то $\phi(W) \le \phi(V) = \im \phi \le W$.
  }
}

\jprf{
  Пусть $\phi \in \CL(V), U \le V$, тогда:

  \jen{
    \item Пусть $P \in F[x]$, тогда, если $U$ является $\phi$-инвариантным, то оно и
    $P(\phi)$-инвариантно.
    \item Пусть $\lambda \in F$, тогда $U$ является $\phi$-инвариантным тогда и
      только тогда, когда $U$ инвариантно относительно $\phi - \lambda$.
  }
}

\knt{
  $\lambda \fuck$
}

\jprf{
  \jen{
    \item Если $U$ $\phi$-инвариантно, то $\phi(U) \le U$, значит $\phi^2(U) =
    \phi(\phi(U)) \le \phi(U) \le U$. Тогда, по тем же соображениям, $\phi^n(U)
    \le U$. А тогда, если $P = p_n x^n + \ldots + p_0$, то $P(\phi)(U) =
    (p_n\phi^n + p_{n - 1} \phi^{n - 1} + \ldots + p_0)(U) \le p_n\phi^n(U) + p_{n
    - 1}\phi^{n - 1}OUP + \ldots + p_0U \le U$.
  \item Из первого пункта, если $U$ $\phi$-инвариантно, то $U$ также $(\phi -
    \lambda)$-инвариантно. Но $\phi = (\phi - \lambda) + \lambda$, а значит
    если $U$ $(\phi - \lambda)$-инвариантно, то $U$ также и $\phi$-инвариантно.
  }
}

\section{Собственные векторы}

Пусть $\phi \in \CL(V), U \le V$ --- $\phi$-инвариантно, $\dim U = 1$. Тогда
$U = \fa{\vc u}, \vc u \neq \vc 0$, и $\phi(\vc u) \in U$, то есть $\phi(\vc u)$

\jdf{
  Пусть $\phi \in \CL(V)$. Вектор $\vc v \in V$ называется собственным вектором
  (СВ) преобразования $\phi$, если

  \jen{
    \item $\vc v \neq \vc 0$.
    \item $\phi(\vc v) = \lambda \vc v, \lambda \in F$. В этом случае $\lambda$
      называется собственным значением оператора $\phi$, соответствующему
      вектору $\vc v$.
  }
}

\jdf{
  Скаляр $\lambda \in \F$ является собственным (СЗ) значением оператора $\phi$,
  если он соответствует хотя бы одному собственному вектору.
}

Если $\vc v$ --- собственный вектор с собственным значением $\lambda$, то
$\phi(\vc v) = \lambda \vc v \iff \phi(\vc v) - \lambda \vc v = \vc 0 \iff (\phi
- \lambda)(\vc v) = \vc 0 \iff \vc v \in \ker (\phi - \lambda)$

\jdf{
  Если $\lambda$ --- собственное значение оператора $\phi$, то подпространство
  $\ker (\phi - \lambda) = V_\lambda$ называется собственным подпространством
  оператора $\phi$.
}

\knt{
  Подпространство $V_\lambda$ состоит в точности из $\vc 0$ и всех собственных
  векторов соответствующих $\lambda$.
}

\knt{
  Иногда используют понятие $V_\lambda$ и для $\lambda$, не являющихся
  собственными значениями, тогда $V_\lambda = 0$.
}

Если $\lambda$ является собственным значением

Пусть $e$ --- базис в $V$, $\phi \is[e] A$. Тогда $\phi - \lambda \is[e] A -
\lambda E$.

\jprp{
  Если $\lambda$ --- собственное значение $\phi$, то $\ker (\phi - \lambda) \neq
  0$, а значит, $\phi - \lambda$ --- вырожденный оператор, то есть его матрица
  $A - \lambda E$ --- вырожденная матрица, а тогда $\fp{A - \lambda E} = 0$.
}

\jdf{
  Пусть $A \in M_n(F)$, тогда ее характеристический многочлен --- это многочлен
  такой, что $\chi_A(x) = \fp{A - xE} =$
  \jeq{ = \fp{\jarr{c c c c}{
        a_{11} - x & a_{12} & a_{13} & \ldots \\
        a_{21}     & a_{22} - x & a_{23} & \ldots \\
        a_{31}     & a_{32} & a_{33} - x & \ldots \\
        \vdots & \vdots & \vdots & \ddots \\
  }}}
}

\jcrl{
  Если $\phi \in \CL(V), \phi \is[e] A$, то $\lambda \in F$ --- собственное
  значение $\phi$ тогда и только тогда, когда $\lambda$ --- корень $\chi_A$.
}

\jprp{
  Пусть $A, B \in M_n(F)$ --- подобные матрицы, то есть $\exists S \in GL_n(F):
  B = S\rev A S$. Тогда $\chi_A = \chi_B$.
}

\jprf{
  $\chi_B = \fp{B - xE} = \fp{X\rev AS - x \cdot S\rev ES} = \fp{S\rev (A -
  xE)S} = \fp{S\rev} \cdot \fp{A - xE} \cdot \fp{S} = \chi_A$.
}

\jcrl{
  Характеристический многочлены матрицы $\phi$ не зависит от выбора базиса.
}

\jdf{
  Этот многочлен называется характеристическим многочленом преобразования $\phi$
  и обозначается $\chi_\phi$.
}

Распишем \hyperref[equation2.2.1]{формулу (2.2.1)}:

\[ \chi_A = \hyperref[equation2.2.1]{(2.2.1)} = (a_{11} - x)(a_{22} -
x)\cdot(a_{nn} - x) \]

\jdf{
  Пусть $A = (a_{ij}) \in M_n(F)$. Тогда ее след --- это $\tr A = \sm{i =
  1}{n}a_{ii}$.
}

\jprp{
  Если $\chi_A = (-1)^n x^n + p_{n - 1} x_{n - 1} + \cdot + p_0$, то $p_{n - 1}
  = (-1)_{n - 1} \tr A$, $p_0 = \det A$.
}

\jprf{
  В формуле полного разложения $\fp{A -xE}$, кроме элементов главной диагонали,
  каждое слагаемое имеет минимум 2 сомножителя не с главной диагонали,
  следовательно, $p_{n - 1}$ -- это коэффициент при $x_{n - 1}$ в $\prd{i =
  1}{n}(a_{ii} - x)$, то есть $p_{n - 1} = \sm{i = 1}{n}a_{ii}(-1)_{n - 1} =
  (-1)_{n - 1} \tr A$.

  $p_0 = \chi_A(0) = \fp{A - oE} = \fp{A}$.
}

\kexc{
  Выразите $p_k$ в терминах миноров матрицы $A$ порядка $n - k$.
}

\jcrl{
  Если $A$ и $B$ --- подобные матрицы, то $\tr A = \tr B$ и $\det A = \det B$.
}

\jdf{
  Если $\phi \in \CL(V)$, то его след $\tr \phi$ --- это след любой его матрицы,
  а его определитель --- это определитель любой его матрицы.
}

\jprp{
  Пусть $\phi \in \CL(V)$, а $\lambda_1, \ldots, \lambda_k$ --- его различные
  собственные значения, тогда собственные подпространства $V_{\lambda_i}$
  образуют прямую сумму:
  \[ V_{\lambda_1} + \ldots + V_{\lambda_k} = V_{\lambda_1} \oplus \ldots
  \oplus V_{\lambda_k} \]
}

\jprf{
  Предположим противное, то есть $\vc 0 = \vc v_1 + \ldots + \vc v_k$, где $\vc
  v_i \in V_{\lambda_i}$, и не все $\vc v_i$ --- нули. Тогда выберем такое
  представление, в котором наибольшее количество нулевых векторов. Выкинув все
  $\lambda_i$, для которых $\vc v_i = 0$ получаем:
  \[ \vc 0 = \vc  v_1 + \ldots + \vc v_k, \vc v_i \in V_{\lambda_i}, \vc v_i
  \neq 0 \]
  Тогда $l > 1$. $\vc 0 = \sm{i = 1}{l} \vc v_i$. Значит, $\vc 0 = \phi(\vc 0) =
  \phi\fd{\sm{i = 1}{l} v_i} = \sm{i = 1}{l} \phi(\vc v_i) = \sm{i = 1}{l}
  \lambda_i \vc v_i$
  $\vc 0 = \sm{i = 1}{l} \lambda_i \vc v_i - \sm{i = 1}{l} \lambda_l \vc v_l =
  \sm{i = 1}{l - 1}(\lambda_i - \lambda_l) \vc v_i$. Таким образом мы получили
  представление нуля в виде суммы $l - 1$ вектора. Противоречие.
}

\jth{
  Пусть $\phi \in \CL(V), \dim V = n$, и пусть $\chi_\phi$ имеет $n$ различных
  корней (в поле $F$). Тогда существует базис $e$ пространства $V$ такой, что
  \fuck $\phi$ в этом базисе диагональна, то есть $\phi \iss \diag(\lambda_1,
  \ldots, \lambda_n) =$
  \[ = \fuck \]
}

\jnt{
  Если $e = (\vc e_1, \ldots, \vc e_n)$ --- базис, то $\phi \is[e]
  \diag(\lambda_1, \ldots, \lambda_n) \iff \phi(e_1) = \lambda_1 e_1 \land
  \phi(e_2) = \lambda_2 e_2 \land \ldots$.
}

\jprf{
  Все $\lambda_i$ являются собственными значениями $\phi_i$ выберем при $r = 1,
  \ldots, n$ собственный вектор $\vc e_i$ с собственным значением $\lambda_i$.
  Тогда $\vc e_i \in V_{\lambda_i}$, поскольку сумма $V_{\lambda_1} + \ldots +
  V_{\lambda_n}$ --- прямая, то есть $(\vc e_1, \ldots, \vc e_n)$ --- линейно
  независимы, а значит это базис. Тогда $\phi \is[e] \diag(\lambda_1, \ldots,
  \lambda_n)$.
}

\jdf{
  Оператор $\phi \in \CL(V)$ называется диагонализуемым, если существует базис
  $e$ пространства $V$ такой, что $\phi \is[e] A$ --- диагональна.
}

\jdf{
  Матрица $M \in M_n(F)$ называется диагонализуемой, если она подобна
  диагональной матрице.
}

\knt{
  В координатах оператор с диагональной матрицей $\diag(\lambda_1, \ldots,
  \lambda_n)$ действует как
  \[ \fd{\jarr{c}{x_1\\ \vdots \\ x_n}} \mapsto A\fd{\jarr{c}{x_1\\ \vdots
  \\ x_n}} = \fd{\jarr{c}{\lambda_1 x_1\\ \vdots \\ \lambda_n x_n}} \]
  Геометрически --- это растяжение вдоль $i$-той оси в $\lambda_i$ раз.
}

\twrap{поворот}{l}{0pt}{
  \tdraw[-Stealth]{(0, 0) coordinate (C)--(3, 1) coordinate (A)}
  \tdraw[-Stealth]{(0, 0)--(-1, 3) coordinate (B)}
  \tp{pic[draw, -Stealth, angle radius=1.4cm] {angle = A--C--B}}
}

\knt{
  Не любой оператор диагонализуем. Например, на $V_2$ рассмотрим оператор
  поворота на $\pi/2$. Он не диагонализуем, что у него нет собственного вектора.
  Его матрица в ОНБ:
  \[ \phi \is A = \fd{\jarr{c c}{0 & -1 \\ 1 & 0 \\}} \]
  А также
  \[ \chi_\phi = \chi_A = \fp{\jarr{c c}{-x & -1 \\ 1 & -x \\}} = x^2 + 1 \]
}

\knt{
  Интерпретируем на матричный язык тот факт, что $V_{\lambda_i}$ образует прямую
  сумму. Пусть $\lambda_1, \ldots, \lambda_k$ --- собственные значения $\phi$,
  тогда
  \[ V_{\lambda_1} \oplus \ldots \oplus V_{\lambda_k} \oplus U = V \]
  В соответствующем базисе имеем
  \[ \phi \is[e] \fd{\jarr{c c c c}{\lambda_1&&&\\&\ddots&&\\&&\lambda_1&\\}} \]
}

\jdf{
  Пусть $\phi \in \CL(V), \lambda$ --- его собственное значение. Тогда

  \jit{
    \item алгебраическая кратность собственного значения $\lambda$ --- это
      кратность его как кратность $\chi_\phi$.
    \item Геометрическая кратность --- это $\dim V_\lambda = \ker (\phi -
      \lambda)$.
  }
}

\jprp{
  Пусть $\phi \in \CL(L), U \le V$ --- $\phi$-инвариантное подпространство.
  Тогда $\psi = \phi|_U \in \CL(U)$, и $\chi_\psi \dv \chi_\phi$.
}

\jprf{
  Пусть $e = (\vc e_1, \ldots, \vc e_n)$ --- базис в $V$ такой, что $(\vc e_1,
  \ldots, \vc e_n)$ --- базис в $U$. Тогда
  \[ \phi \is[e] \fd{\jarr{c|c}{A & b \\ \hline 0 & c \\}} \]
  где $\psi \iss[(\vc e_1, \ldots, \vc e_n)] A$.
  Тогда $\chi_\phi = \chi_D = \fp{D -xE} =$
  \[ = \fp{\jarr{c:c}{A - xE & B \\ \hdashline 0 & C - xE \\}} = \]
  $= \fp{A - xE} \cdot \fp{C - xE} = \chi_A \cdot \chi_C = \chi_\psi \cdot
  \chi_C$.
}

\jth{
  Если $\lambda$ --- собственное значение $\phi$, то его алгебраическая
  кратность не меньше геометрической кратности.
}

\jprf{
  Пусть $U = V_\lambda = \ker (\phi - \lambda)$ --- инвариантно относительно
  $\phi$. Значит, $\chi_\phi$ делится на $\chi_{\phi|_U}$, но $\phi|_U = \lambda
  \is \lambda E$. Тогда, если $\dim U = k$, то $\chi_{\phi|_U} = \fp{\lambda E -
  x E} = (\lambda - x)^k \dv \chi_\phi$. Значит, кратность корня $\lambda$ у
  $\chi_\phi \ge k$.
}

\jnt{
  Геометрическая кратность $\lambda$ --- это $\dim \ker (\phi - \lambda)$. Если
  $\phi \is[e] A$, то $\phi - \lambda \is[e] A - \lambda E$, и $\dim \ker (\phi
  - \lambda) - \dim V - \dim \im (\phi - \lambda) = n - \rk (A - \lambda E)$.
}

\kexm{
  Если
  \[ A = J_n = \fuck \]
}
