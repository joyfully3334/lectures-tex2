\chapter{Многочлены}

\lectureinfo{1}{04.02.2026}

\section{Корни многочленов}

\krmd{\
  \jit{
    \item Многочлен над полем F единственным образом представляется в виде 
    $P = p_0 + p_1x + \ldots + p_nx^n$, причем единственным образом.
    Все многочлены над полем $F$ образуют кольцо, обозначаемое $F[x]$,
    а также алгебру над $F$.
    \item Базис $F[x]$ --- это $\fd{1, x, x^2, \ldots}$
    \item Значения многочлена: Если $A$ --- произвольная алгебра над $F$, $a \in A$, то
    $P(a) = p_0 \cdot 1 + p_1 \cdot a + p_2 \cdot a^2 + \ldots + p_n a^n \in A$
    \item $P(a) + Q(a) = (P + Q)(a)$
    \item $(PQ)(a) = P(a) \cdot Q(a)$
  }
}

\krmd{\
  \jit{
    \item $A$ делим на $B \neq 0$:
      \[ A = QB + R, \deg R < \deg B \]

    \item У многочленов $A$ и $B$ существует наибольший общий делитель:
      \[ \NOD(P, Q) = AP + BQ, A, B \in F[x] \]
    \item Основная теорема арифметики: любой ненулевой многочлен единственным
      образом раскладывается в произведение неприводимых многочленов.
  }
}

\jdf[Корень многочлена]{
  Пусть $D \in F[x]$, ($F$ --- поле), $a \in F$. Тогда a --- корень многочлена
  $P$, если $P(a) = 0$.
}

\jth[Безу]{
  Пусть $P \in F[x]; a \in F$. Тогда $a$ --- корень $P$ тогда и только тогда,
  когда $(x - a) \dv P$.
}

\jprf{
  Разделим $P$ на $x - a$ с остатком.
  $D = Q \cdot (x - a) + R$, где $\deg R < \deg (x - a) = 1$, то есть $R$ ---
  константа. Подставим $a$ в $P$: $P(a) = Q(a) \cdot (a - a) + R = R$.
  $a$ --- корень $P \iff P(a) = 0 \iff R = 0 \iff (x - a) \dv P$.
}

\knt{
  В любом случае $R = P(a)$.
}

\jdf[Кратность корня]{
  Пусть $a$ --- корень многочлена $P \in F[x]: P \neq 0$. Его кратность --- это наибольшее
  натуральное число $k$ такое, что $(x - a)^k \dv P$.
}

\jth[О сумме кратностей корней]{
  Пусть $P \in F[x]: P \neq 0, \deg P = n$. Тогда сумма кратностей всех его
  корней, не превосходит $n$.
}

\jprf{
  Пусть $a_1, \ldots, a_d$ --- корни $P$, $k_1, \ldots, k_d$ --- их кратности,
  тогда $(x - a_i)^{k_i} \dv P$. Но $(x - a_i)$ и $(x - a_j)$ взаимно просты при
  различный $a_i \neq a_j$. Поэтому $Q := \prd{i = 1}{a} (x - a_i)^{k_i} \dv P$.
  Значит, $\deg Q = \sm{i = 1}{d} k_i \le \deg P = n$.
}

\knt{
  Если $\sm{i}{} k_i = n$, то это означает, что $P = \alpha \prd{i = 1}{d}
  (x - a_i), \alpha \in F^*$
}

\jdf[Линейно факторизуемый многочлен]{
  Такой многочлен называется линейно факторизуемым.
}

\jth[Основная теорема алгебры]{
  Любой многочлен над полем комплексных чисел линейно факторизуем.
}

\knt{
  Поля с таким же свойством называются алгебраически замкнутыми.
}

\jdf[Кратный корень]{
  Корень $a \in F$ многочлена $P \in F[x]$ называется кратным корнем, если
  его кратность $> 1$, иначе он называется простым.
}

\jdf[Формальная производная многочлена]{
  Пусть $P \in F[x]: P = p_0 + p_1 x + \ldots + p_n x^n = \sm{i}{} p_i x^i$.
  Его формальной производной называется $P' = p_1 + 2p_2 x + \ldots + n p_n x^{n
  - 1} = \sm{i \ge 1}{} i p_i x^{i - 1}$
}

\jprp[Свойства формальной производной]{\
  \jen{
    \item $\fd{\alpha P + \beta Q}' = \alpha \cdot P' + \beta \cdot Q',\quad
      \alpha, \beta \in F,\quad P, Q \in F[x]$
    \item $\fd{PQ}' = P'Q + PQ'$ и, более того, $\fd{P_1 \ldots P_n}' = 
      P_1'P_2\ldots P_n + P_1 P_2'P_3\ldots P_n + \ldots + P_1 \ldots P_{n - 1}
      P_n'$
    \item $\fd{P(Q)}' = P'(Q) \cdot Q'$
  }
}

\jprf{\
  \jen{
  \item Если $P = \sm{i}{} p_i x^i, Q = \sm{i}{} q_i x^i$, то
    $(\alpha P + \beta Q)' = \fd{\sm{i}{}\fd{\alpha p_i + \beta q_i}x^i}' =
    \sm{i}{} (\alpha p_i + \beta q_i) \cdot i \cdot x^{i - 1} =
    \alpha \cdot \sm{i}{} i p_i x^{i - 1} + \beta \cdot \sm{i}{} i q_i x^{i - 1}
    = \alpha P' + \beta Q'$.

    \knt{
      Мы доказали, что взятие производной от многочлена --- это линейное
      преобразование $\phi: F[x] \to F[x]$.
    }
  \item При фиксированном многочлене $Q$ обе части равенства являются линейными
    операторами, зависящими от $P$. Линейный оператор однозначно задается
    значениями этого оператора на базисе, следовательно достаточно проверить
    равенство для $P = x^n, n \ge 0$. Аналогично, достаточно рассмотреть
    случай, когда $Q = x^m, m \ge 0$.
    \[ P'Q + Q'P = n x^{n - 1} \cdot x^m + m x^{m - 1} \cdot x^n = (n + m)
      x^{n + m - 1} = (x^{n + m})' = (PQ)' \]
    Равенство $(P_1, P_2, \ldots, P_n)' = ...$ доказывается индукцией по $n$.
    База при $n = 2$ уже доказана. Переход: $(P_1, \ldots, P_{n - 1}, P_n)
    = (P_1, \ldots, P_{n - 1})'P_n +
    P_1\ldots P_{n - 1}P_n' = P_1'P_2\ldots P_n + \ldots + P_1 \ldots P_{n - 1}'
    P_n + P_1 \ldots P_{n - 1}P_n'$.
  \item Опять же, левая и правая части линейны по $P$, а значит достаточно
    проверить равенство при $P = x^n$. Тогда $\fd{P(Q)}' = \fd{Q^n}' =
    nQ^{n + 1}Q' = P'(Q) \cdot Q' $
  }
}

\jth[Условия кратности корня]{
  Пусть $P \in F[x], a \in F$.

  \jen{
    \item $a$ является кратным корнем многочлена $P$ тогда и только тогда, когда
      $P(a) = P'(a) = 0$.
    \item Если $a$ --- корень кратности $\ge k$, то $P(a) = P'(a) = \ldots =
      P^{(k-1)}(a) = 0$.
  \item Если $P(a) = \ldots = P^{(k - 1)}(a) = 0$, то $a$ --- корень $p$
    кратности $\ge k$ при условии, что $\Char F = 0$ или $\Char F \ge k$.
  }
}

\jprf{
  Пусть $t$ --- кратность корня $a$, то есть, $P = (x - a)^t Q$, где
  $Q(a) \neq 0$. Тогда $P' = ((x - a)^t)'Q + (x - a)^t Q' = t(x - a)^{t - 1} Q +
  (x - a)^t Q' = (x - a)^{t - 1}(tQ + (x - a)Q')$.

  \jen{
    \item Если $t = 1$, то $P'(a) = tQ(a) + 0 = Q(a) \neq 0$.
      Если $t \ge 2$, to $P'(a) = 0$.
    \item Если $a$ --- корень кратности $\ge k$ в $P$, то $a$ --- корень кратности
      $\ge k - 1$ в $P' \implies \ldots \implies a$ --- корень кратности $\ge 1$
      в $P^{(k - 1)}$.

      \knt{
        Подставляя во вторую скобку $a$, получаем $tQ(a) + 0 = tQ(a)$, что будет
        нулем в случае, если $p \dv t$.
      }
    \item Заметим, что при $\Char F = 0$ или $t < \Char F$, корень $a$
      многочлена $P'$ имеет кратность $t - 1$, так как $tQ(a) \neq 0$.
      Значит, $a$ --- корень $P$ кратности $t \implies a$ --- корень
      $P'$ кратности $t - 1 \implies a$ --- корень $P''$ кратности $t - 2
      \implies \ldots \implies a$ --- корень $P^{(t)}$ кратности 0, то есть,
      не корень. Таким образом, если $\Char F = 0$ или $k \le \Char F$, то
      случай $t < k$ невозможен, иначе $P^{(t)}(a) \neq 0$. Поэтому $t \ge k$.
  }
}

\kexm{
  Рассмотрим многочлен $Q = x^p - 1 \in \Z_p[x]$. У него есть
  корень $a = 1$ кратности $\le p$. С другой стороны, $Q' = px^{p - 1} = 0 = Q''
  = Q''' = \ldots$ Таким образом, $Q(1) = Q'(1) = Q''(1) = \ldots = 0$.
  Значит третье утверждение применимо прим $k = p$, следовательно, $1$ ---
  корень $Q$ кратности $\ge p$. Значит, $Q = \alpha(x - 1)^p = (x - 1)^p$.
}

\knt{
  С некоторыми изменениями, тот же метод работает для выяснения, на какую
  степень неприводимого многочлена $Q$ делится $P$.
}

\chapter{Линейные преобразования}

\section{Инвариантные подпространства}

\krmd{\
  \jit{
    \item Линейное преобразование пространства $V_i$ --- это линейное
      отображение $\phi: V \to V$, то есть такое, что

      \jen{
        \item $\phi(\vc v_1 + \vc v_2) = \phi(\vc v_1) + \phi(\vc v_2)$
        \item $\phi(\alpha \vc v) = \alpha\cdot \phi(\vc v), \quad \alpha \in F$
      }

      (Считаем, что $V$ --- конечномерное пространство над полем $F$)

    \item Если $e = (\vc e_1, \ldots, \vc e_n)$ --- базис в $V$, то $\phi$
      однозначно задается своей матрицей в базисе $\phi \is[e] A$.
      Причем такой, что $\phi(e) = (\phi(\vc e_1), \ldots, \phi(\vc e_n)) = eA$.
      Если $\phi \is[e] A, \vc v \is[e] \alpha$, то $\phi(\vc v) \is[e] A\alpha$.

    \item $\CL (V)$ --- множество всех линейных преобразований $V$ --- линейное
      пространство над $F$, а также кольцо, то есть алгебра над $F$. Для
      фиксированного базиса $e$ сопоставление $\phi \is[e] A$ дает изоморфизм
      алгебр $\CL(V) \cong M_n(F)$.

    \item Если $e, e'$ --- базисы в $V$, $e' = eS$, причем $\phi \is[e] A, \phi
      \is[e] A'$, то $A' = S\rev A S$.

    \item Матрицы $A, A' \in M_n(F)$ подобны, если $\exists S \in GL_n(F): A' =
      S\rev A S$.
  }
}

\jdf[Инвариантное подпространство]{
  Пусть $V$ --- линейное пространство над $F$, $\phi \in \CL(V), U \le V$.
  Подпространство $U$ называется инвариантным относительно $\phi$ (или
  $\phi$-инвариантным), если $\phi(U) \subseteq U$ (то есть, $\forall \vc u \in
  U \hk \phi(\vc u) \in U$).
}

\knt{
  Это свойство достаточно проверять для базиса в $U$.
}

\krmd{\
  \jit{
    \item Если $U$ подпространство в $V$, то $\phi(U)$ тоже подпространство в
      $V$.
    \item Образ линейного оператора: $\phi \in \CL(V) \implies \im \phi =
      \phi(V) \le V$
    \item Ядро линейного оператора: $\Ker \phi = \phi\rev(\vc 0) \le V$
  }
}

\knt{
  Если $U$ --- $\phi$-инвариантное подпространство, то $\phi|_U \in \CL(U)$.
}

\jprp{
  Пусть $\phi \in \CL(V)$, $U \le V$, и $e = (\vc e_1, \ldots, \vc e_n)$ ---
  базис в $V$ такой, что его префикс $(\vc e_1, \ldots, \vc e_k)$ --- базис в
  $U$. Тогда $U$ --- $\phi$-инвариантное подпространство $V$ тогда и только
  тогда, когда
  \[ \phi \is[e] A = \fd{\jarr{c:c}{B & C \\ \hdashline 0 & D \\}} \]
}

\jprf{
  $U$ --- $\phi$-инвариантное подпространство, значит $\forall \vc u \in U \hk
  \phi(\vc u) \in U$, а значит, $\forall i=1,\ldots,k \hk \phi(\vc e_i) \in U$,
  то есть, $\forall i=1,\ldots,k \hk \phi(\vc e_i) \in \fa{\vc e_1, \ldots,
  \vc e_k}$. Отсюда получаем требуемое.
}

\jnt{
  Если $U$ --- $\phi$ инвариантно и
  \[ \phi \is[e] \fd{\jarr{c:c}{B & C \\ \hdashline 0 & D \\}} \]
  то $\phi|_U \iss[(\vc e_1, \ldots, \vc e_k)] B$.
}

\jnt{
  Если $V = U_1 \oplus U_2$, где $U_1, U_2$ --- $\phi$-инвариантны, то в
  базисе, согласованным с $U_1$ и $U_2$, имеем
  \[ \phi \is[e] A = \fd{\jarr{c:c}{B & 0 \\ \hdashline 0 & C \\}} \]
}

\jprp{
  Пусть $U_1, U_2$ --- $\phi$-инвариантные подпространства, тогда $U_1 \cap U_2$
  и $U_1 + U_2$ также $\phi$-инвариантны.
}

\jprf{\
  \jen{
    \item $\vc u \in U_1 \cap U_2 \implies \phi(\vc u) \in U_1 \cap U_2$
    \item Если $\vc u \in U_1 + U_2$, то $\vc u = \vc u_1 + \vc u_2 \implies
      \phi(\vc u) = \phi(\vc u_1) + \phi(\vc u_2) \in U_1 + U_2$
  }
}

\jprp{
  Пусть $\phi, \psi \in \CL(V)$, причем $\phi\psi = \psi\phi$. Тогда $\Ker \psi$
  и $\im \psi$ инвариантны относительно $\phi$.
}

\jprf{
  Пусть $\vc u \in \Ker \psi$, то есть, $\psi(\vc u) = \vc 0$. Необходимо
  проверить, что $\phi(\vc u) \in \Ker \psi$, то есть, $\psi(\phi(\vc u)) =
  \psi\phi(\vc u) = \phi\psi(\vc u) = \phi(\psi(\vc u)) = \phi(\vc 0) = \vc 0$.

  Пусть теперь $\vc u \in \im \psi$, то есть, $\vc u = \psi(\vc v), \vc v \in V$.
  Тогда $\phi(\vc u) = \phi(\psi(\vc v)) = \psi(\phi(\vc v)) \in \im \psi$
}

\knt{
  Мы будет применять это утверждение в случае $\phi = P(\psi)$, где $P$
  --- это некоторый многочлен, ведь любой такой многочлен коммутирует с $\phi$.
}
