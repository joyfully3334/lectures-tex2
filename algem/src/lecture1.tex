\chapter{Многочлены}

\section{Многочлены}

\tk[overlay]{\node (n1){};}
Конспект лекции пока не доделан, формулы и
определения могут быть с ошибками (вообще говоря, и так могли, но здесь
вероятность наткнуться на них сильно выше). Значок \fuck означает пропуск.
\tk[overlay]{\node (n2){};}

\tk[overlay]{
  \tp[fill=red, opacity=0.2]{
      ($(current page.west|-n1)+(20mm - 2mm, 14pt)$)--
      ($(current page.east|-n1)+(-20mm + 2mm, 14pt)$)--
      ($(current page.east|-n2)+(-20mm + 2mm, -6pt)$)--
      ($(current page.west|-n2)+(20mm - 2mm, -6pt)$)--cycle
    }
}

\lectureinfo{1}{04.02.2026}

\krmd{
  Многочлен над полем F единственным образом представляется в виде 
  $P = p_0 + p_1x + \ldots + p_nx^n$. Они образуют кольцо, обозначаемым $F[x]$

  Базис $F[x]$ --- это $\fd{1, x, x^2, \ldots}$
}

\krmd{
  Значения многочлена: Если $A$ --- алгебра над $F$, то
  $P(a) = p_0 \cdot 1 + p_1 \cdot a + p_2 \cdot a^2 + \ldots + p_n a^n \in A$

  $P(a) + Q(a) = (p + \fuck)$
  \fuck
}

\krmd{
  Делимость, деление с остатком

  $A \fuck$ на $B \fuck$

  $A = QB + R, \deg R = \deg B$

  $\NOD(P, Q) = AP + BQ, A, B \in F[x]$

  Основная теорема арифметики.
}

\jdf{
  Пусть $D \in F[x]$, ($F$ -- поле), $a \in F$. Тогда a -- корень многочлена
  $P$, если $P(a) = 0$.
}

\jth[Безу]{
  Пусть $P \in F[x]; a \in F$. Тогда $a$ -- корень $P$ тогда и только тогда,
  когда $x - a \dv P$.
}

\jprf{
  Разделим $P$ на $x - a$ с остатком.
  $D = Q \cdot (x - a) + R$, где $\deg R < \deg (x - a) = 1$, то есть $R$ --
  константа. Подставим $a$ в $P$: $P(a) = Q(a) \cdot (a - a) + R = R$.
  $a$ -- корень $P \iff P(a) = a \iff R = 0 \iff (x - a) \dv P$.
}

\knt{
  И любом случае $R = P(0)$.
}

\jdf{
  Пусть $a$ -- корень многочлена $P \in F[x]: P \neq 0$. Его кратность --- это наибольшее
  натуральное число $k$ такое, что $(x - k)^k \dv P$.
}

\jth{
  Пусть $P \in F[x]: P \neq 0, \deg P = n$. Тогда сумма кратностей всех его
  корней, не превосходит $n$.
}

% TODO: Fix indices: a_1, ..., a_???, and other
\jprf{
  Пусть $a_1, \ldots, a_k$ -- корни $P$, $k_1, \ldots, k_k$ -- их кратности,
  тогда $(x - a_i)_{k_i} \dv P$. Но $x - a_i, x - a_j (a_i \neq a_j)$ взаимно
  просты. Поэтому $\prd{i = 1}{a} (x - a_i)^{k_i} \dv P$.

  \fuck

  $\deg Q \subseteq \deg P = n$

  $\deg Q = \sm{}{}{}$
}

\knt{
  Если $2k_i = n$, то это означает, что $P = \alpha \prd{i = 1}{d} (x - a_i),
  \alpha \in Ft$
}

\jdf{
  Такой многочлен называется линейно факторизуемым.
}

\jth[Основная теорема Алгебра]{
  Любой многочлен над полем комплексных чисел линейно факторизуем.
}

\knt{
  Такие поля называется алгебраически замкнутыми.
}

\jdf{
  Корень $a \in F$ многочлена $P \in F[x]$ называется кратным корнем, если
  его кратность $> 1$, иначе он называется простым.
}

\jdf[Формальная производная]{
  Пусть $P \in F[x], P = p_0 + p_1 x + \ldots + p_n x^n = \sm{i}{} p_i x^i$.
  Его формальной производной называется $P' = p_1 + 2p_2 x + \ldots + n p_n x^{n
  - 1} = \sm{i \ge 1}{} i p_i x^{i - 1}$
}

\jprp[Линейность формальной производной]{\
  \jen{
    \item $\fd{\alpha + \beta Q}' = \alpha \cdot P' + \beta \cdot Q',
      \alpha, \beta \in F, P, Q \in F[x]$
    \item $\fd{PQ}' = P'Q + PQ'$ и, более того, $\fd{p_1, \ldots, p_n}' = 
      p_1'p_2\ldots p_n + p_1 p_2'p_3\ldots p_n + \ldots + p_1 \ldots p_{n - 1}
      p_n'$
    \item $\fd{P(Q)}' = P'(Q) \cdot Q'$
  }
}

\jprf{\
  \jen{
  \item Если $P = \sm{i}{} p_i x^i, Q = \sm{i}{} q_i x^i$, то
    $(\alpha P + \beta Q)' = \fd{\sm{i}{}\fd{\alpha p_i + \beta q_i}x^i}' =
    \sm{i}{} i(\alpha p_i + \beta q_i)x^{i - 1} =
    \alpha \cdot \sm{i}{} i p_i x^{i - 1} + \beta \cdot \sm{i}{} i q_i x^{i - 1}
    = \alpha P' + \beta Q'$.
  }
  \item При фиксированном \fuck части расв... \fuck --- линейным операторами от
    $P$. Линейном однозначно задается значениями на базисе $\implies$ достаточно
    проверить для $P = x^n, n \ge 0$. Аналогично, достаточно рассмотреть случай,
    когда $Q = x^m, m \ge 0$.

    $P'Q + Q'P = n x^{n - 1} \cdot x^m + m x^{m - 1} \cdot x^n = (n + m) x^{n + m
    - 1} = (x^{n + m})' = (PQ)'$.

    Равенство $(P_1, P_2, \ldots, P_n)' = ...$ доказывается индукцией по $n$.
    База при $n = z$ \fuck
    Для перехода: $(P_1, \ldots, P_{n - 1}, P_n) = (P_1, \ldots, P_{n - 1})'P_n +
    P_1\ldots P_{n - 1}P_n' = P_1'P_2\ldots P_n + \ldots + P_1 \ldots P_{n - 1}'
    P_n + P_1 \ldots P_{n - 1}P_n'$.
  \item \fuck левая и правая части .. по $P \implies$ достаточно проверить
    равенство при $P = x^n$. Тогда $\fd{P(Q)}' = \fd{Q^n}' = nQ^{n + 1}Q' =
    P'(Q) \cdot Q' $
}

\jth{
  Пусть $P \in F[x], a \in F$.

  \jen{
    \item $a$ является кратным корнем многочлена $P$ тогда и только тогда, когда
      $P(a) = P'(a) = 0$.
    \item Если $a$ -- корень кратности $\ge k$, то $P(a) = P'(a) = \ldots =
      P^{(\fuck)}(a) = 0$.
  \item Если $P(a) = \ldots = P^{(n - 1)}(a) = 0$, то $a$ -- корень $p$
    кратности $\ge k$ при условии, что $\Char F = 0$ или $\Char F \ge k$.
  }
}

\jprf{
  Пусть $t$ -- кратность корня $a$, то есть, $P = (x - a)^t Q$, где $Q(a) \neq 0$.
  Тогда $P' = ((x - a)^t)Q + (x - a)^t Q' = (x - a)^{t - 1} Q + (X -a)^t Q' =
  (x - a)^{t - 1}(tQ + (x - a)Q')$.

  \jen{
    \item Если $t = :$, то $P'(a) = tQ(a) + 0 = Q(a) \neq 0$.
      Если $t \ge 2$, to $P'(a) = 0$.
    \item $a$ -- корень кратности $\ge k$ в $P \implies a$ -- корень кратности
      $\ge k - 1$ в $P' \implies \ldots \implies a$ -- корень кратности $\ge 1$
      в $P^{(k - 1)}$.

      \jnt{
        Подставляя во вторую скобку $a$, получаем $tQ(a) + 0 = tQ(a) \neq 0$
      }
    \item Заметим, что при $\sm{}{} \Char F = 0$ или $t < \Char F$, корень $a$
      многочлена $P'$ имеет кратность $t - q: tQ(a) \neq 0$. Значит, $a$ --
      корень $P$ кратности $t \implies a$ -- корень $P'$ кратности $t - 1
      \implies a$ -- корень $P''$ кратности $t - 2 \implies \ldots \implies a$
      -- корень $P^{(t)}$ кратности 0, то есть, не корень. Тишли образом, если
      $\Char F = 0$ или $k \le \Char F$, то случай $t < k$ невозможен:
      $P^(t)(a) \neq 0$. Поэтому $t \ge k$.
  }
}

\jexm{
  При $F = \Z_p$. Рассмотрим многочлен $Q = x^p - q \in \Z_p[x]$. У него есть
  корень $a = 1$ кратности $\le p$. С другой стороны, $Q' = px^{p - 1} = 0 = Q''
  = Q''' = \ldots$. Таким образом, $Q(q) = Q'(q) = Q''(1) = \ldots = 0$.
  Значит третье утверждение применимо прим $k = p$, следовательно, $1$ -- корень
  $Q$ кратности $\ge p$. Значит, $Q = \alpha(x - 1)^p = (x - 1)^p$.
}

\jnt{
  С некоторыми изменениями, тот же метод работает для выяснения на какую степень
  неприводимого многочлена $Q$ делится $P$.
}

\chapter{Линейные преобразования}

\jdf{
  Линейное преобразование простраства $V_i$ --- это линейное отображение $\phi:
  V \to V$.
}

\fuck

(V - )

$e$ --- базис в $V$

$\phi \is[e] A$ --- матрица $\phi$ в базисе $e$:
$\phi(e) = (\phi(\vc e_1), \ldots, \phi(\vc e_n)) = eA$

Если $\phi \is[e] A, \vc v \is[e] \alpha$, то $\phi(\vc v) \is[e] A\alpha$.

$\CL (V)$ --- множество всех линейных преобразований $V$ --- линейное
пространство над $F$, а также кольцо, то есть алгебра над $F$. Для
фиксированного базиса $e$ сопоставления $\phi \is[e] A$ дает изоморфизм алгебр
$\CL(V) \cong M_n(F)$.

Если $e, e'$ --- базисы в $V$, $e' = eS$, причем $\phi \is[e] A, \phi \is[e]
A'$, то $A' = S\rev A S$.

\jrmd{
  Матрицы $A, A' \in M_n(F)$, если $\exists S \in GL_n(F): A' = S\rev A S$.
}

\jdf[Инвариантное подпространство]{
  Пусть $V$ --- линейное пространство над $F$, $\phi \in \CL(V), U \le V$.
  Подпространство $U$ называется инвариантным относительно $\phi$ (или
  $\phi$-инвариантным), если $\phi(U) \subseteq U$ (то есть, $\forall \vc u \in
  U \hk \phi(\vc u) \in U$).
}

\jnt{
  Это евклидово \fuck.
}

\jrmd{
  Если $U$ подпространство в $V$, то $\phi(U)$ тоже подпространство в $V$.
}

$\phi \in \CL(V) \implies \im \phi = \phi(V) \le V$

$\Ker \phi = \phi\rev(\vc 0) \le V$

\jnt{
  Если $U$ -- $\phi$-инвариантное подпространство, то $\phi|_u \in \CL(U)$.
}

\jprp{
  Пусть $\phi \in \CL(V)$, $U \le V$, и $e = (\vc e_1, \ldots, \vc e_n)$ --
  базис в $V$ такой, что его префикс $(\vc e_1, \ldots, \vc e_k)$ -- базис в
  $V$. Тогда $U$ -- $\phi$-инвариантное подпространство $V$ тогда и только
  тогда, когда
  \[ \phi \is[e] A = \fd{\jarr{c:c}{B & C \\ \hdashline 0 & D}} \]
}

\jprf{
  $U$ -- $\phi$-инвариантное подпространство $\iff \forall \vc u \in U \hk
  \phi(\vc u) \in U \iff \forall i=1,\ldots,k \hk \phi(\vc e_i) \in U \iff
  \forall i=1,\ldots,k \hk \phi(\vc e_i) \in \fa{\vc e_1, \ldots, \vc e_k}$.
}

\jnt{
  Если $U$ -- $\phi$ инвариантно и
  \[ \phi \is[e] \fd{\jarr{c:c}{B & C \\ \hdashline 0 & D}} \]
  то $\phi|_u \is[(\vc e_1, \ldots, \vc e_k)] B$.
}

\jnt{
  Если $V = U_1 \oplus U_2$, где $U_1, U_2$ --- $\phi$-инварианатны, то в
  базисе, согласованным с $U_1$ и $U_2$, имеем
  \[ \phi \is[e] A = \fd{\jarr{c:c}{B & 0 \\ \hdashline 0 & C}} \]
}

\jprp{
  Пусть $U_1, U_2$ --- $\phi$-инвариантные подпространства, тогда $U_1 \cap U_2$
  и $U_1 + U_2$ также $\phi$-инвариантны.
}

\jprf{\
  \jen{
    \item $\vc u \in U_1 \cap U_2 \implies \phi(\vc u)$
    \item Если $\vc u \in U_1 + U_2$, то $\vc u = \vc u_1 + \vc u_2 \implies
      \phi(\vc u) = \phi(\vc u_1) + \phi(\vc u_2) \in U_1 + U_2$
  }
}

\jprp{
  Пусть $\phi, \psi \in \CL(V)$, причем $\phi\psi = \psi\phi$. Тогда $\Ker \psi$
  и $\im \psi$ инвариантны относительно $\phi$.
}

\jprf{
  Пусть $\vc u \in \Ker \psi$, то есть, $\psi(\vc u) = \vc 0$. Необходимо
  проверить, что $\phi(\vc u) \in \Ker \psi$, то есть, $\psi(\phi(\vc u)) =
  \psi\phi(\vc u) = \phi(\vc 0) = \vc 0$.

  Пусть $\vc u \in \im \psi$, то есть, $\vc u = \psi(\vc u), \vc v \in V$.
  Тогда $\phi(\vc u) = \phi(\psi(\vc v)) = \psi(\phi(\vc v))$
}

\jnt{
  \fuck \fuck \fuck
}
