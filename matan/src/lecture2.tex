\chapter{Дифференциальное исчисление функций многих переменных}

\section{Предел и непрерывность}

\lectureinfo{2}{06.02.2026}

\jdf[Функция многих переменных]{
  Функцией многих переменных называем функцию $f: D \to \R$, где
  $D \subset \R^n$.
}

\jdf[Предел функции многих переменных]{
  Пусть $\vc x_0$ --- предельная точка $D$. Тогда предел определяется:

  \jit{
    \item по Коши:
    \[ \dlm{\vc x \to  \vc x_0}{\vc x \in D} f(\vc x) = l \iff \forall
      \epsilon > 0 \then \exists \delta > 0: \forall \vc x \in D \land
      0 < \rho_x(\vc x, \vc x_0) < \delta \hk \rho_y(f(\vc x), l) < \epsilon \]
  \item по Гейне:
    \[ \dlm{\vc x \to  \vc x_0}{\vc x \in D} f(\vc x) = l \iff
      \forall \fc{\vc x_m} \subset D: \lm[m \to \infty] \vc x_m = \vc x_0
      \land \vc x_m \neq \vc x_0 \hk \lm[m \to \infty] f(\vc x_m) = l \]
  }
}

\jdf[Предел по совокупности переменных]{
  Если $\vc x_0$ --- внутренняя точка $D \cup \fc{\vc x_0}$, то $\lm[\vc x
  \to \vc x_0] f(\vc x)$ называется пределом по совокупности переменных.
}

\jdf[Предел по направлению]{
  Если $D = D_{\vc d}$ --- луч с началом в точке $\vc x_0$ в направлении вектора $\vc d$ (или
  его пересечение с проколотой окрестностью точки $\vc x_0$), то предел
  $\dlm{\vc x \to \vc x_0}{\vc x \in D_{\vc d}} f(\vc x)$ называется пределом
  по направлению $\vc d$.
}

\jlm{
  \[ \dlm{\vc x \to \vc x_0}{\vc x \in D_{\vc d}} f(\vc x) =
  \lm[t \to 0+] f(\vc x_0 + t\vc d) \]
}

\jprf{
  Достаточно заметить, что
  \[ \vc x \in D_{\vc d} \iff \vc x = \vc x_0 + t \vc d,\ t \ge 0\ (0 < t < r) \]
}

\jth{
  Если существует предел $\lm[\vc x \to \vc x_0] f(\vc x)$ по совокупности
  переменных, то существует предел $\dlm{\vc x \to \vc x_0}{\vc x \in D_{\vc d}}
  f(\vc x)$ по любому направлению $\vc d \neq 0$. Обратное, вообще говоря, не
  верно.
}

\jprf{
  Из определения по Коши, если $\vc x \in D_{\vc d}, 0 < \fp{\vc x - \vc x_0}
  < \delta$, то $\vc x \in D, 0 < \fp{\vc x - \vc x_0} < \delta$.
}

\jexm{
  \[ f(x, y) = \left\{\jarr{c c}{\frac{xy^2}{x^2 + y^2}, & (x, y) \neq (0, 0)
  \\ 0, & (x, y) = (0, 0)}\right. \]
  По \hyperref[lemma6.1.1]{лемме 1} получаем
  \[ \dlm{(x, y) \to (0, 0)}{(x, y) \in D_{(\alpha, \beta)}} f(x, y) =
    \lm[t \to 0+] f(0 + t\alpha, 0 + t\beta) =
  \lm[t \to 0+]\frac{t^3\alpha\beta^2}{t^2\alpha^2 + t^4\beta^4} = 0 \]
  По определению Гейне возьмем последовательность $\fd{\frac{1}{m^2}, \frac{1}{m}}
  \to (0, 0)$, тогда
  \[ f\fd{\frac{1}{m^2}, \frac{1}{m}} = \frac{1/m^4}{2/m^4} = \frac{1}{2} \]
  Тогда общего предела у функции нет, хотя по каждому направлению --- есть.
}

\jdf[Повторный предел]{
  \[ \lm[x \to x_0] \fd{\lm[y \to y_0] f(x, y)} \]
}

\jnt{
  Наличие предела по совокупности переменных и повторных пределов никак не
  связано.
}

\jexm{
  \[ g(x, y) = \left\{\jarr{c c}{\sqrt{x^2 + y^2} \sin \frac{1}{x}
  \sin \frac{1}{y}, & (x, y) \neq (0, 0) \\
  0, & (x, y) = (0, 0) \\}\right. \]
  У данной последовательности есть предел по совокупности
  \[ \lm[(x, y)\to(0, 0)] g(x, y) = 0 \text{, так как } \fp{g(x, y)} \le
  \fp{(x, y)} \]
  $\lm[y \to 0]{g(x, y)} \text{ существует и равен 0} \iff x = 0
  \text{ или } x = \textstyle\frac{1}{k}, k \in K$. Тогда внутренний предел не определен в
  проколотой окрестности, а значит повторного предела действительно нет.
}

Если $\vc x_0$ --- предельная точка области определения $D(f)$, то $f$ непрерывна
в $\vc x_0$ тогда и только тогда, когда $\dlm{\vc x \to \vc x_0}{\vc x \in D(f)}
f(\vc x) = f(\vc x_0)$. Если $\vc x_0$ --- изолированная точка $D(f)$, то $f$
непрерывна в $\vc x_0$.

Непрерывность по совокупности переменных и
непрерывность по направлению определяются естественным образом.

\jth[Непрерывность сложной функции многих переменных]{
  Если $f: D \to \R$ непрерывна в точке $\vc y_0 = \fd{y_0\supr 1, \ldots, y_0\supr n} \in D$,
  функции $g_j(\vc x), j = 1, \ldots, n$ непрерывны в точке $\vc x_0 \in G
  \subset \R^m, g_j(\vc x_0) = y_0\supr j, j = 1, \ldots, n$, то сложная функция
  $h(\vc x) = f\fd{g_1(\vc x), \ldots, g_n(\vc x)}$ непрерывна в $\vc x_0$.
}

\jprf{
  Возьмем последовательность Гейне:
  \kmlt{\fc{\vc x_k} \subset D(h): \lm[k \to \infty] \vc x_k = \vc x_0 \hk
  \lm[k \to \infty] g_j(\vc x_k) = g_j(\vc x_0) = y_0^{(j)} \implies \\
  \implies \fd{g_1(\vc x_k), \ldots, g_n(\vc x_k)} \tto[k \to \infty] \vc y_0
  \implies f(g_1(\vc x_k), \ldots, g_n(\vc x_k)) \tto[k \to \infty] f(\vc y_0)}
}

\jth[Кантор]{
  Если $f$ непрерывна на компактном множестве $K \subset \R^n$,
  то $f$ равномерно непрерывна на $K$.
}

\jprf{
  (От противного). Определение равномерной непрерывности на $K$:
  \[ \forall \epsilon > 0 \then \exists \delta > 0: \forall \vc x_1, \vc x_2 \in
  K: \fp{\vc x_1 - \vc x_2} < \delta \hk \fp{f(\vc x_1) - f(\vc x_2)} < \epsilon \]
  Тогда отрицанием будет
  \[ \exists \epsilon > 0 : \forall \delta > 0 \then \exists \vc x_1, \vc
  x_2 \in K: \fp{\vc x_1 - \vc x_2} < \delta \hk \fp{f(\vc x_1) - f(\vc x_2)} \ge
  \epsilon \]
  Построим две последовательности по $\delta := 1, \frac{1}{2}, \ldots$
  \[ \fc{\vc x_{1, k}}, \fc{\vc x_{2, k}} \subset K: \fp{\vc x_{1, k} - \vc
  x_{2, k}} < \frac{1}{k} \hk \fp{f(\vc x_{1, k}) - f(\vc x_{2, k})} \ge
  \epsilon \]
  По критерию компактности $\exists \fc{\vc x_{1, k_j}}: \lm[j \to \infty]
  \vc x_{1, k_j} = \vc x_0 \in K$, тогда
  \[ \fp{\vc x_{2, k_j} - \vc x_0} \le \fp{\vc x_{2, k_j} - \vc x_{1, k_j}} +
  \fp{\vc x_{1, k_j} - \vc x_0} \implies
  \lm[j \to \infty] \vc x_{2, k_j} = \vc x_0\]
  Так как $f$ непрерывна в $\vc x_0$, то
  \[ \lm[j \to \infty] f(\vc x_{1, k_j}) = \lm[j \to \infty] f(\vc x_{2,
  k_j}) = f(\vc x_0) \]
  \[ \fp{f(\vc x_{1, k_j}) - f(\vc x_{2, k_j})} \ge \epsilon \]
  Но вместе они не могут выполняться. Противоречие.
}

\jdf[Линейная связность]{
  Множество $A$ в $\R^n$ называется линейно связным, если $\forall \vc x \neq
  \vc y \in A$ существует кривая $\gamma_i: \fb{a, b} \to A
  \fd{\gamma_i \subset A}$ такая, что $\gamma(a) = \vc x, \gamma(b) = \vc y$.
}

\jdf[Связность]{
  Метрическое пространство $X$ называется связным, если его нельзя представить
  объединением двух непересекающихся непустых открытых множеств.
}

\jprp{
  Множество $A$ в $\R^n$ является связным, если его нельзя представить
  объединением двух непересекающихся непустых открытых в нем множеств.

  Множество $G \subset A$ открыто в $A \iff \exists \CG \subset \R^n$
  открытое, такое, что $G = \CG \cap A$.
}

\jprf{
  $G$ открыто в $A$, значит $\forall x \in G \then\exists U_\epsilon^A(\vc x)
  \subset G$. Возьмем $\CG := \bigcup\limits_{\vc x \in G} U_\epsilon(\vc x)$.
  Тогда $G = \CG \cap A$.
}

\jnt{
  \[ U_\epsilon^A(\vc x) := \fc{\vc y \in A: \fp{\vc x - \vc y} < \epsilon} \]
}

\jth{
  Если $A \in \R^n$ линейно связно, то оно связно.
}

\jprf{
  (От противного). Если $A$ --- не связно, то найдутся $G_1, G_2$ ---
  открытые множества в $A$, такие, что
  \[ G_1 \neq \varnothing, G_2 \neq \varnothing,\quad G_1 \cap G_2 = \varnothing
  \quad G_1 \cup G_2 = A\]
  Пусть $\vc x \in G_1, \vc y \in G_2$. $A$ --- линейно связно, следовательно
  \[ \exists \gamma: [a, b] \to A: \gamma(a) = \vc x, \gamma(b) = \vc y \]
  Обозначим $T := \sup\fc{t: \gamma(t) \in G_1}$, $T \in [a, b]$. Предположим,
  что $\gamma(T) \in G_1$, тогда $\exists G_1 = \CG_1 \cap A$. Так как
  $\gamma(T) \in \CG_1$, то, $U_\epsilon(\gamma(T)) \subset \CG_1$.
  $T$ точно меньше $b$, а значит, $\exists \delta > 0: \gamma([T, T + \delta))
  \subset U_\epsilon(\gamma(T))$. То есть, $\gamma(T)$ не лежит в $G_1$.
  Аналогично доказывается, что $\gamma(T)$ не лежит в $G_2$, и отсюда получаем
  противоречие.
}

\jnt{
  Обратное, вообще говоря, не верно.

  \jexm{
    Контрпример иллюстрирован на \hyperref[figure6.1.1]{рис. 6.1}. Данное
    множество очевидно является связным, и в то же время не является линейно
    связным.
  }
}

\tpic{Контрпример}{
  \tdraw[-Stealth]{(-2, 0)--(2.5, 0) node[below left] {$x$}}
  \tdraw[-Stealth]{(0, -1.5)--(0, 2) node[below left] {$y$}}
  \tp{(0, 0) node[below left] {$O$}}
  \tdraw{(2 * 3.14 * 23:1 + 1 / 23 / 2 / 3.14 * 100)
  foreach \i in {23, 23.5, ..., 400}{
    --(2 * 3.14 * \i:1 + 1 / \i / 2 / 3.14 * 100)
  }}
  \tp{(4.5, 0) node {
    $
    \left\{\jald{
      &x = r \cos \phi \\
      &y = r \sin \phi \\
      &r = 1 + \frac{1}{\phi} \\
    }
    \right.
    $
  }}
}
