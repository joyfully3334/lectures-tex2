\chapter{Дифференциальное исчисление функций многих переменных}

\section{Предел и непрерывность}

\lectureinfo{2}{06.02.2026}

$f: D \to \R = Y$

$D \subset \R^n = X$

Приведем определение предела:

\jdf[Предел]{
  Пусть $\vc x_0$ -- предельная точка $D$. Тогда предел определяется:

  по Коши:

  \[ \lm[] f(\vc x) = l \iff \forall \epsilon > 0 \then \exists \delta > 0 :
  \forall \vc x \in D, 0 < \rho^x(\vc x, \vc x_0) < \delta \hk \rho_y(f(\vc x),
l) < \epsilon \]

  по Гейне:

  \[ \lm[] f(\vc x) = l \iff \forall \fc{\vc x_m} \subset D: \lm[m \to
  \infty] \vc x_m = \vc x_0, \vc x_m \neq \vc x_0 \hk \lm[m \to \infty] f(\vc
  x_m) = l \]

  Если $\vc x_0$ -- внутренняя точка D, то $\lm[\vc x \to \vc x_0]$ называется
  пределом по совокупности переменных.

  Если $D = D_{\vc d}$ -- луч с началом в точке $\vc x_0$ в направлении вектора $\vc d$ (или
  его пересечение с проколотой окрестностью точки $\vc x_0$), то предел
  $\lm[\vc x \to \vc x_0] f(\vc x)$ называется пределом по направлению $\vc d$
}

\jlm{
  \[ \lm[\vc x \to \vc x_0] f(\vc x) = \lm[t \to 0+] f(\vc x_0 + t\vc d) \]

  \[ \vc x \in D_{\vc d} \iff \vc x = \vc x_0 + t \vc d, t \ge 0 (0 < t < r) \]
}

\jth{
  Если существует предел по совокупности переменных
  $\lm[\vc x \to \vc x_0] f(\vc x)$, то существует предел по любому
  направлению $\lm[\vc x \to \vc x_0]$. Обратное, вообще говоря не верно.
}

\jprf{
  Если $\vc x \in D_{\vc d}, 0 < \fp{\vc x - \vc x_0} < \delta$, то
  $\vc x \in D, 0 < \fp{\vc x - \vc x_0} < \delta$.
}

\jexm{
  \[ f(x, y) = \left\{\jarr{c c}{\frac{xy^2}{x^2 + y^2}, & (x, y) \neq (0, 0)
  \\ 0, & (x, y) = (0, 0)}\right. \]

  \[ \lm[(x, y) \to (0, 0)] f(x, y) = \lm[t \to 0+] f(0 + t\alpha, 0 +
  t\beta) = \lm[t \to 0+]\frac{t^3\alpha\beta^2}{t^2\alpha^2 + t^4\beta^4} = 0 \]

  По определению Гейне возьмем последовательность $\fd{\frac{1}{m^2}, \frac{1}{m}}
  to (0, 0)$, тогда

  \[ f\fd{\frac{1}{m^2}, \frac{1}{m}} = \frac{1/m^4}{2/m^4} = \frac{1}{2} \]

  Тогда общего предела у функции нет, хотя по каждому направлению -- есть.
}

\jdf[Повторный предел]{
  \[ \lm[x \to x_0] \fd{\lm[y \to y_0] f(x, y)} \]
}

\knt{
  Наличие предела по совокупности переменных и повторных пределов никак не
  связано.

  \jexm{
    \[ g(x, y) = \left\{\jarr{c}{\sqrt{x^2 + y^2} \sin \frac{1}{x} \sin \frac{1}{y}\\
    0}\right. \]

    \[ \lm[(x, y)\to(0, 0)] g(x, y) = 0 \text{, так как } \fp{g(x, y)} \le
    \fp{(x, y)} \]

    \[ \lm[y \to 0]{g(x, y)} \text{ существует и равен 0} \iff x = 0
    \text{ или } x = \frac{1}{k}, k \in K \]
  }
}

Если $\vc x_0$ -- предельная точка больати определения $D(f)$, то $f$ непрерывна
в $\vc x_0$ тогда и только тогда, когда $\lm[x \to \vc x_0] f(\vc x) = f(\vc
x_0)$

Если $\vc x_0$ -- изолированная точка $D(f)$, то $f$ непрерывна в $\vc x_0$.
Непрерывность по совокупности переменных и непрерывность по направлению
определяются естественным образом.

\jth[Непрерывность сложной функции многих переменных]{
  Если $f: D \to \R$ непрерывна в точке $\vc y_0 = \fd{y_0^{(1)}, \ldots, y_0^{(n)}} \in D$,
  функции $g_j(\vc x), j = 1, \ldots, n$ непрерывны в точке $\vc x_0 \in G
  \subset \R^m, g_j(\vc x_0) = y_0^{(j)}, j = 1, \ldots, m$, то сложная функция
  $h(\vc x) = f\fd{g_1(\vc x), \ldots, g_n(\vc x)}$ непрерывна в $\vc x_0$.
}

\jprf{
  Возьмем последовательность Гейне:

  \[ \fc{\vc x_k} \subset D(h): \lm[k \to \infty] \vc x_k = \vc x_0 \hk
  \lm[k \to \infty] g_j(\vc x_k) = g_k(\vc x_0) = y_0^{(j)} \implies
  \fd{g_1(\vc x_k), \ldots, g_n(\vc x_k)} \to[k \to \infty] \vc y_0 \implies
  f \fuck \]
}

\jth[Кантор]{
  Если $f$ непрерывна на компактном множестве $K \subset \R^n$,
  то $f$ равномерно непрерывна на $K$.
}

\jprf{
  (От противного). Определение равномерной непрерывности на $K$:

  \[ \forall \epsilon > 0 \then \exists \delta > 0: \forall \vc x_1, \vc x_2 \in
  K, \fp{\vc x_1 - \vc x_2} < \delta \hk \fp{f(\vc x_1) - f(\vc x_2)} < \epsilon \]

  Тогда отрицанием будет

  \[ \exists \epsilon > 0 \then \forall \delta > 0 \then \exists \vc x_1, \vc
  x_2 \in K, \fp{\vc x_1, \vc x_2} < \delta: \fp{f(\vc x_1) - f(\vc x_2)} \ge
  \epsilon \]

  Построим последовательность по $\delta := 1, \frac{1}{2}, \ldots$

  \fuck
  \[ \fc{\vc x_{1, k}}, \fc{\vc x_{2, k}} \subset K \]

  По критерий компактности $\exists \fc{{\vc x_{1, k}}_j}, \lm[j \to \infty]
  {\vc x_{1, k}}_j = \vc x_0 \in K$

  Тогда

\[ \fp{{\vc x_{2, k}}_j - \vc x_0} \le \fp{{\vc x_{2, k}}_j - {\vc x_{1, k}}_j} +
  \fp{{\vc x_{1, k}}_j - \vc x_0} \implies
  \lm[j \to \infty] {\vc x_{2, k}}_j = \vc x_0\]

  Так как $f$ непрерывна в $\vc x_0$, то

  \[ \lm[j \to \infty] f({\vc x_{1, k}}_j) = \lm[j \to \infty] f({\vc x_{2,
  k}}_j) = f(\vc x_0) \]

  \[ \fp{f(\vc x_{1, k}}_j) - f({\vc x_{2, k}}_j) \ge \epsilon \]

  Но вместе они не могут выполняться. Противоречие.
}

\jdf[Линейная связность]{
  Множество $A$ в $\R^n$ называется линейно связным, если $\forall \vc x \neq
  \vc y \in A \then\exists$ кривая $\gamma_i \subset A$ такая, что
  $\gamma(a) = \vc x, \gamma(b) = \vc y$.
}

\jdf[Связность]{
  Метрическое пространство $X$ называется связным, если его нельзя представить
  объединением двух непересекающихся непустых открытых множеств.
}

\jprp{
  Множество $A$ в $\R^n$ является связным, если его нельзя представить
  объединиением двух непересекающихся непустых открытых в нем множеств.

  Множество $G \subset A$ открыто в $A \iff \exists \mathfrak g \subset \R^n$
  открытое, такое, что $G = \mathfrak g \cap A$.
}

\jprf{
  \[ \forall x \in G \then\exists U_\epsilon^A(\vc x) \subset G \]

  Возьмем

  \[ \mathfrak g := \bigcup_{\vc x \in G} U_\epsilon(\vc x) \]
}

\knt{
  \[ U_\epsilon^A(\vc x) := \fc{\vc y \in A: \fp{\vc x - \vc y} < \epsilon} \]
}

\jth{
  Если $A \in \R^n$ линейно связно, то оно линейно связно.
}

\jprf{
  (От противного). $A$ -- не связно, следовательно $\exists G_1, G_2$ ---
  открытых множества в $A$, таких, что

  \[ G_1 \neq \varnothing, G_2 \neq \varnothing, G_1 \cap G_2 = A \]

  Пусть $\vc x \in G_1, \vc y \in G_2$. $A$ -- линейно связно, следовательно

  \[ \exists \gamma \subset A, \gamma(a) = \vc x, \gamma(b) = \vc y \]

  $\gamma: [a, b] \to A$

  Обозначим $T := \sup\fc{t: \gamma(t) \in G_1}$, $T \in [a, b]$. Предположим,
  что $\gamma(T) \in G_1$, тогда $\exists G_1 = \mathfrak g_1 \cap A$

  \[ \gamma(T) \in \mathfrak g_1 \implies U_\epsilon(\gamma(T)) \subset
  \mathfrak g_1 \]

  \[ T < b \implies \exists \delta > 0: \gamma([T, T + \delta)) \subset
  U_\epsilon(\gamma(T)) \]

  Получаем противоречие.
}

\jnt{
  Обратное, вообще говоря, не верно.

  \jexm{
    Контрпример иллюстрирован на \hyperref[figure6.1.1]{рис. 6.1}:
    \[ x = r \cos \phi \]
    \[ y = r \sin \phi \]
    \[ r = 1 + \frac{1}{\phi} \]
  }
}

\tpic{Контрпример}{
  \tdraw{(0, 0) circle (1)}
  \tdraw[-Stealth]{(0, 0)--(3, 0) node[below] {x}}
}
