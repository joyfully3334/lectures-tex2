\setcounter{chapter}{4}

\chapter{Неопределенный интеграл}

\setcounter{section}{2}

\section{Интегрирование иррациональных и трансцендентных функций}

\lectureinfo{1}{04.02.2026}

\jdf{
  $R(u, v, \ldots, w)$ --- рациональная функция, если
  \[ R(u, v, \ldots, w) = \frac{P(u, v, \ldots, w)}{{Q(u, v, \ldots, w)}} \]
  где $P, Q$ --- алгебраические многочлены.
}

Рассмотрим некоторые случаи интегрирования рациональных функций:

\jen{
  \item
    \[ \int R\fd{x, \fd{\frac{ax + b}{cx + d}}^r\,}dx \]
    где $R$ --- рациональная функция, $ad - bc  \neq 0$, $r \in Q$

    тогда $r = \frac{p}{q}, p \in \Z, q \in \N$

  \[ \frac{ax + b}{cx + d} = t^q \iff (ax + b)) - (cx + d)t^q = 0 \iff
  x = \frac{dt^q - b}{a - ct^q}\]

  \[ \fd{\frac{ax + b}{cx + d}}^q = t^p \]
  Подвид:
    \[ \int R\fd{x, \fd{\frac{ax + b}{cx + d}}^r, \ldots, \fd{\frac{ax + b}{cx + d}}^{r_n}} \]

  \item 
    \[ \int R\fd{x, \sqrt{ax^2 + bx + c}}dx \]

    где $a \neq 0$

    Подстановки Эйлера:

    \jen{
      \item $a > 0$

      \[ \sqrt{ax^2 + bx + c} = \pm \sqrt a \pm t \]
      \[ ax^2 + bx + c = ax^2 \pm 2 \sqrt a x t + t^2 \]

      Тогда

      \[ x = \frac{t^2 - c}{b \mp 2 \sqrt a t} \]

      $x'$ -- рациональная функция от $t$.
    \item $c > 0$
      \[ \sqrt{ax^2 + bx + c} = \pm x t \pm \sqrt c \]
      \[ ax^2 + bx + c = x^2 t^2 \pm 2xt \sqrt c + c \]
      \[ x = \frac{-b \pm 2t \sqrt c}{a - t^2} \]

      Отсюда $x'$ -- рациональная функция
      \item
        \[ ax^2 + bx + c = a(x - x_1)(x - x_2) \]
        где $x_1 \neq x_2$

        \[ \sqrt{ax^2 + bx + c} = \pm \sqrt a (x - x_1)t \]
        Возведем в квадрат обе части:
        \[ a(x - x_1)(x - x_2) = \fp{a}(x - x_1)^2 t^2 \]
        Тогда, сокращая, получим:
        \[ x = \frac{ax_2 - \fp{a}x_1 t^2}{a - \fp{a}t^2} \]
        Значит, $x'$ -- рациональная функция
    }
    \item
      \[ \int R\fd{x, \sqrt{ax^2 + bx + c}}dx \]

      Введем $Y = ax_2 + bx + c, y = \sqrt Y$, тогда

      \[ R(x, y) = \frac{R_1(x) + R_2(x) y}{R_3(x) + R_4(x) y} \]

      Затем мы можем (?) записать выражение через многочлены $P$:

      \[ R(x, y) = \frac{P_1(x) + P_2(x) y}{P_3(x) + P_4(x) y} = \]
      \[ = \frac{\fd{P_1(x) + P_2(x)y}\fd{P_3(x) - P_r(x)y}}{P_3^2(x) -
      P_4^2(x)Y} = \]
      \[ = R_1(x) + R_2(x)y \]

      \[ \int P(x)ydx = \int \frac{P(x)Y}{y} dx \]

      Хотим найти

      \[ \int \frac{P(x)}{y} dx \]

      Пусть $\deg P = n$, тогда интеграл запишется в виде:

      \[ \int \frac{P(x)}{y} dx = Q(x)y + \lambda \int \frac{dx}{y} \]

      $\deg Q = n - 1, \lambda \in \R$

      После дифференцирования получаем:

    \[ \frac{P(x)}{y} = Q'(x)y + \frac{Q(x)Y'}{2 \sqrt Y} + \frac{\lambda}{y} = \]

    \[ = \frac{Q'(x) \cdot Y + \frac{1}{2} Q(x)Y' + \lambda}{y} \]

    Тогда нам необходимо найти интеграл вида:

    \[ V_m = \int \frac{x^m}{y} dx \]

    Продифференцируем:

    \[ \fd{x_{m - 1}y}' = (m - 1)x^{m - 2}y + \frac{x^{m - 1}\fd{2ax + b}}{2y} \]

    Тогда

    \[ x^{m -1}y = (m - 1) \int \frac{x^{m - 2}(ax_2 + bx + c)}{y}dx +
    \frac{1}{2} \int \frac{2ax^m + bx^{m - 1}}{y} dx = \]

    \[ a(m - 1)V_m + (m - \frac{1}{2})bV_{m - 1} + c(m - 1)V_{m - 2} \]

    Итого получаем:

    \[ V_m = p_{m - 1}(x)y + \lambda \int \frac{dx}{y} \]

    После получим интегрирования слагаемые:

    \[ \int \frac{A}{x - \alpha}^k \frac{dx}{y} \]

    Введем замену $x - \alpha = \frac{1}{t}$

    \[ dx = - \frac{dt}{t^2} \]

    \[ y = \sqrt{a\fd{\frac{1}{t} + \alpha}^2 + b\fd{\frac{1}{t} + \alpha} + c}
    = \frac{\widetilde y}{\fp{t}} \]

    Итого

    \[ \int \frac{A}{x - \alpha}^k \frac{dx}{y} = - \int \frac{A t^k \fp{t}}{t^2
    \widetilde y}dt \]

  \item Дифференциальный бином.
    \[ \int x^m(a + bx^n)^p dx \]
    где $m, n, p \in Q$

    Этот интеграл берется только в трех случаях:

    \jen{
      \item $p \in \Z, \nu$ --- НОК знаменателей m, n
        Возьмем $x = t^\nu$, тогда $dx = \nu t^{\nu - 1}dt$

        Пусть теперь $p \ni \Z$, тогда сделаем замену $x^n = z$, откуда
        $x = z^{\frac{1}{n}}; dx = \frac{1}{n} z^{\frac{1}{n} - 1} dz$

        Тогда интеграл получится в виде
        \[ \frac{1}{n} \int z^{^frac{m + 1}{n} - 1}\fd{a + bz}^p dz \]

        Обозначим рациональное $q := \frac{m + 1}{n} - 1$, тогда

        \[ \frac{1}{n} \int z^q (a + bz)^p dz \]
      \item $q \in \Z$
        $a + bz = t^\mu$, где $\mu$ --- знаменатель $p$
        \[ z = \frac{1}{b}\fd{t^\mu - a} \]
        \[ dz = \frac{1}{b}\mu t^{\mu - 1}dt \]
      \item $p + q \in \Z$
        \[ \int z^q(a + bz)^p dz = \int z^{p + q}\fd{\frac{a + bz}{z}}^p dz \]

        Введем замену: $\frac{a + bz}{z} = t^\mu$

        \[ z = \frac{a}{t^\mu - b} \]
    }
  \item \jth[Абеля]{
      Пусть $R$ --- многочлен степени $2g + 2$ без кратных корней. Если существуют
      многочлены $P, Q \neq 0$ такие, что $P^2 - Q^2 R = 1$, то найдется
      многочлен $r$ степени $g$ такой, что
      \[ \int \frac{r}{\sqrt R}dx \]
      выражается в элементарных функциях.
    }

    \jprf{
      \[ \fd{\ln \frac{P + Q \sqrt R}{P - Q \sqrt R}} = \]
      \[ = \frac{P - Q\sqrt R}{P + Q\sqrt R} \cdot
    \frac{\fd{P + Q'\sqrt R + \frac{QR'}{2\sqrt R}}\fd{P - Q\sqrt R} - \fd{P' - Q'\sqrt R
    - \frac{QR'}{2\sqrt R}}\fd{P + Q\sqrt R}}{\fd{P - Q\sqrt R}^2} =\]
    \[ = \frac{-4P'QR + 4Q'P + 2QR'P}{2\sqrt R} =
    \frac{P\fd{QR' + 2Q'R} - 2P'QR}{\sqrt R} =
  \frac{2P^2P' - 2P'Q^2R}{Q\sqrt R} = \frac{2r}{\sqrt R}\]
    Продифференцируем $P^2 - Q^2R = 1$:
    \[ 2PP' - 2QQ'R - Q^2R' = 0 \implies 2PP' = Q\fd{2Q'R + QR'} \implies P' =
    Q_2 \]
    $\deg P' = \deg Q + g$
    }
  \item
    \[ \int R \fd{\sin x, \cos x}dx \]
    Универсальная подстановка: $\tg \frac{x}{2} = t$
}
