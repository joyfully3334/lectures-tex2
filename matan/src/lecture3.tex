\lectureinfo{3}{11.02.2026}

\jlm{
  Множество $E \subset \R$ связно тогда и только тогда, когда $E$ --- промежуток.
}

\jprf{
  По \hyperref[theorem6.1.4]{теореме 4}, из того, что $E$ промежуток следует,
  что $E$ связно. Тогда пусть $E$ не промежуток. Тогда
  \[ \exists x_1, x_2 \in E: [x_1, x_2] \not \subset E \implies \exists c \in
  (x_1, x_2): c \notin E \]
  Тогда построим два открытых множества в $E$: $G_1 = (-\infty, c) \cap E, G_2 =
  (c, +\infty) \cap E$. Данные множества открыты в $E$, непусты и непересекаются
  между собой, и $G_1 \cup G_2 = E$, следовательно $E$ не связно.
}

\jlm{
  Если $X$ --- связное метрическое пространство, $f: X \to Y$ непрерывно, то
  $f(X)$ связно.
}

\jprf{
  (От противного). Представим $f(X) = G_1 \cup G_2$, где $G_1 = \CG_i \cap f(X),
  G_i \neq \varnothing, i = 1, 2; G_1 \cap G_2 = \varnothing$. Тогда
  $f\rev(\CG_1) \cup f\rev(\CG_2) = X$. Противоречие.
}

\jth[Больцано-Коши в $\R^n$]{
  Если $E \subset \R^n$ связно, $f: E \to \R$ непрерывна, то
  \[ \forall A, B: \exists \vc a, \vc b \in E, f(\vc a) = A, f(\vc b) = B
  \then\forall \gamma \in (A, B) \hk \exists \vc c \in E: f(\vc c) = \gamma \]
}

\jprf{
  По \hyperref[lemma6.1.3]{лемме 3}, $f(E)$ связно, тогда по
  \hyperref[lemma6.1.2]{лемме 2} $f(E)$ --- промежуток. В таком случае
  требуемое следует из определения промежутка.
}

\jth{
  Каждое открытое связное множество в $\R^n$ линейно связно.
}

\jprf{
  (От противного). Пусть $G$ --- открытое и связное множество. Тогда если $G$ не
  линейно связное, то $\exists \vc x_1, \vc x_2 \in G$ такие, что их нельзя
  соединить кривой в $G$. Введем следующие множества:
  \[  G_1 := \fc{\vc x \in G: \vc x \text{ нельзя соединить с $\vc x_1$ кривой в
  $G$}} \]
  \[ G_2 = G \setminus G_1 \]
  Докажем, что $G_2$ открыто в $G$. Пусть $\vc x_0 \in G_2 \subset G$ тогда
  $\exists \epsilon > 0: U_\epsilon(\vc x_0) \subset G$. Отсюда $U_\epsilon(\vc
  x_0) \subset G_2$. Теперь докажем, что $G_1$ открыто. От противного, $\exists
  \vc x_0 \in G_1 \then\forall \epsilon > 0 \hk U_\epsilon(\vc x_0) \cap G_2
  \neq \varnothing$. $\exists
  \epsilon_1 > 0: U_{\epsilon_1}(\vc x_0) \subset G$.
  $\vc y \in U_{\epsilon_1}(\vc x_0) \cap G_2$. Противоречие.
}

\begin{figure}[ht]
    \centering
  \begin{subfigure}{0.4\textwidth}
    \centering
    \tk[draw]{
      \tp[fill=black!10]{(1, .5) circle (1)}
      \tp[fill]{(-1, -1) circle (.03) node[above left] {$\vc x_1$}}
      \tp[fill]{(1, .5) circle (.03) node[below right] {$\vc x_0$}}
      \tp[draw, -Stealth]{(-1, -1)--(1, .5)}
    }
    \caption{Пример 1}
  \end{subfigure}
  \begin{subfigure}{0.4\textwidth}
    \centering
    \tk{
      \tp[fill=black!10]{(1, .5) circle (1)}
      \tp[fill]{(-1, -1) circle (.03) node[above left] {$\vc x_1$}}
      \tp[fill]{(1, .5) circle (.03) node[below right] {$\vc x_0$}}
      \tp[fill]{($(1, .5)+(-.6, .6)$) circle (.03) node[above left] {$\vc y_0$}}
      \tp[draw, -Stealth]{(-1, -1)--(1, .5)}
      \tp[draw, -Stealth]{(1, .5)--($(1, .5)+(-.6, .6)$)}
    }
    \caption{Пример 2}
  \end{subfigure}
  \caption{Пример на $\R^2$ для \hyperref[theorem6.1.6]{теоремы 6}}
\end{figure}

\section{Дифференцируемость функций многих переменных}

\jdf{
  Пусть $f$ --- действительно-значная функция, определенная в некоторой
  окрестности точки $\vc x_0 \in \R^n$.
  Тогда $f$ называется дифференцируемой в точке $\vc x_0$, если
  $\exists \vc A \in \R^n$ Полное приращение $f$ в точке $\vc x_0$
  $\Delta y := f(\vc x_0 + \vc{\Delta x}) - f(\vc x_0)$ представимо в виде
  $\Delta y = (\vc A, \vc{\Delta x}) + o(\fp{\vc{\Delta x}}), \vc{\Delta x} \to
  \vc 0$. $\vc {\Delta x}$ называется вектором приращений аргументов, $(\vc A,
  \vc{\Delta x}) = dy = df$ называется дифференциалом функции $f$, $\vc A =
  \grad f$ называется градиентом $f$. Вектор приращений аргументов
  --- независимых переменных счиатется вектором дифференциалов этих переменных,
  $\vc{\Delta x} = \vc{dx}$. $dy = (\vc A, \vc{dx})$
}

\jdf{
  Частичным (частным) приращением функции $f$ в точке $\vc x_0 = (x_{1, 0},
  \ldots, x_{n, 0})$ по переменной $x_j$ называется $\dt_j f = f(\vc x_0 +
  \vc{\dt_j x}) = f(\vc x_0)$, где $\vc{\dt_j x} = (0, \ldots, 0, \dt x_j, 0,
  \ldots, 0), j = 1, \ldots, n$. $\dt_j f$ является приращением функции
  $\phi_j(x_j) := f(x_{1, 0}, \ldots, x_{j - 1, 0}, x_j, x_{j + 1, 0}, \ldots,
  x_{n, 0})$ $(\dt_j f = \dt \phi_j(x_{j, 0}))$.
}

\jdf[Частная производная]{
  Частной производной функции $f$ в точке $\vc x_0$ по переменной $x_j$
  называется производная (если она существует) функции $\phi_j$ в точке $x_{j,
  0}$.
  Обозначается:
  \[ \frac{\pal f}{\pal x_j} (\vc x_0) := \frac{d\phi_j}{dx_j} (x_{j, 0}) \]
}

\jth{
  Если $f$ дифференцируема в точке $\vc x_0$, то она имеет частные производные в
  этой точке по всем переменным $x_j, j = 1, \ldots, n$, причем, $\grad f(\vc
  x_0) = \fd{\frac{\pal f}{\pal x_1}(\vc x_0), \ldots,
  \frac{\pal f}{\pal x_1}(\vc x_0)}$. Обратное, вообще говоря, не верно.
}

\jprf{
  Пусть $\vc{\dt x} := \vc{\$dt_j x}$, тогда $\dt y = f(\vc x_0 + \vc{\dt_j x})
  - f(\vc x_0) = \dt_j f = \dt \phi_j (x_{j, 0}) = (A, \vc{\dt_j x}) + o(\fp{\vc
  {\dt_j x}}) = A_j\dt x_j + o(\fp{\dt x_j}), \fp{\vc{\dt_j x}} = \fp{\dt x_j}
  \tto 0 \iff \dt x_j \tto 0 \implies \phi_j$ дифференцируема в точке $x_{j, 0}$,
  $\phi_j(x_{j, 0}) = A_j \implies A_j = \frac{\pal f}{\pal x_j} (\vc x_0)$.
}

\jexm{
  \[ f(x, y) = \left\{\jarr{c c}{\frac{xy}{x^2 + y^2}, & (x, y) \neq (0, 0) \\
  0, & (x, y) = (0, 0)}\right. \]
  $\vc x_0 = (0, 0)$
  \[ \frac{\pal f}{\pal x}(0, 0) = \lm[\dt x \to 0] \frac{f(\dt x, 0) - f(0,
  0)}{\dt x} = 0 \]
  \[ \dt y = \dt f = \frac{\dt x \cdot \dt y}{(\dt x)^2 + (\dt y)^2}
  \overset{?}{=} o(\fp{(\dt x, \dt y}), (\dt x, \dt y) \tto 0 \]
  Возьмем вектор $(\dt x, \dt x) \tto (0, 0)$ при $\dt x \to 0$. Тогда
  \[ \dt f = \frac{\dt x \cdot \dt x}{(\dt x)^2 + (\dt x)^2} = \frac{1}{2}
  \neq 0 (\dt x) \]
}

\jexm{
  \[ g(x, y) = \sqrt{\fp{xy}} \]
  \[ \frac{\pal g}{\pal x}(0, 0) = \lm[\dt x \to 0] \frac{g(\dt x 0) - g(0, 0)}
  {\dt x} = 0 \]
  $\dt g = \sqrt{\fp{\dt x \cdot \dt y}}$
  \[ \left\{\jald{\dt x = r \cos \phi \\ \dt y = r \sin \phi}\right.
  \ \  \fd{r = \sqrt{(\dt x)^2 + (\dt y)^2} = \fp{(\dt x, \dt y)}}  \]
  \[ 0 \le \dt g = r \sqrt{\fp{\cos \phi \sin \phi}} \le r \]
  $\dt g \neq o(r)$
}

\jth{
  Если $f$ дифференцируема в $\vc x_0$, то она непрерывна в $\vc x_0$
}

\jprf{
  Хотим доказать, что $\dt y \to 0$ при $\vc{\dt x} \to \vc 0$. Применим
  неравенство Коши-Буняковского-Шварца:
  \[ \fp{\fd{\vc A, \vc{\dt x}}} \le \fp{\vc A} \cdot \fp{\vc{\dt x}} \]
}

\jth[Достаточное условие дифференцируемости]{
  Если в некоторой окрестности точки $\vc x_0$ $f$ имеет частные производные по
  всем переменным, непрерывные в $\vc x_0$, то $f$ дифференцируема в $\vc x_0$.
}

\jprf{
  $\dt y = f(x_{1, 0} + \dt x_1, \ldots, x_{n, 0} + \dt x_n) -
  f(x_{1, 0}, x_{2, 0} + \dt x_2, \ldots, x_{n, 0} + \dt x_n) +
  f(x_{1, 0}, x_{2, 0} + \dt x_2, \ldots, x_{n, 0} + \dt x_n) -
  f(x_{1, 0}, x_{2, 0}, x_{3, 0} + \dt x_3 \ldots, x_{n, 0} + \dt x_n) +
  f(x_{1, 0}, x_{2, 0}, x_{3, 0} + \dt x_3, \ldots, x_{n, 0} + \dt x_n) -
  \ldots +
  f(x_{1, 0}, x_{2, 0}, \ldots, x_{n - 1, 0}, x_{n, 0} + \dt x_n) -
  f(x_{1, 0}, \ldots, x_{n, 0}) =
  \frac{\pal f}{\pal x_1} (\xi_1, x_{2, 0} + \dt x_2, \ldots, x_{n, 0} + \dt
  x_n) \cdot \dt x_1 + \frac{\pal f}{\pal x_2}(x_{1, 0}, \xi_2, x_{3, 0} + \dt
  x_3, \ldots, x_{n, 0} + \dt x_n) \cdot \dt x_2 + \cdot +
  \frac{\pal f}{\pal x_n}(x_{1, 0}, \ldots, x_{n - 1, 0}, \xi_n) \cdot x_n$
}

\tpic{Эээ}{
  \tp[fill=black!10]{(0, 0) circle (1.4)}
  \tp{(0, 0) node {$\vc x_0$}}
}
