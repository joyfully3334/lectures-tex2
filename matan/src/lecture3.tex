\lectureinfo{3}{11.02.2026}

\jlm{
  Множество $E \subset \R$ связно тогда и только тогда, когда $E$ --- промежуток.
}

\jprf{
  По \hyperref[theorem6.1.4]{теореме 4}, из того, что $E$ промежуток следует,
  что $E$ линейно связно, а значит и связно. Пусть $E$ не промежуток, тогда
  \[ \exists x_1, x_2 \in E: [x_1, x_2] \not \subset E \land \exists c \in
  (x_1, x_2) \setminus E \]
  Тогда построим два открытых множества в $E$: $G_1 = (-\infty, c) \cap E, G_2 =
  (c, +\infty) \cap E$. Данные множества открыты в $E$, непусты, непересекаются
  между собой и $G_1 \cup G_2 = E$, следовательно $E$ не связно.
}

\jlm{
  Если $X$ --- связное метрическое пространство, $f: X \to Y$ непрерывно, то
  $f(X)$ связно.
}

\jprf{
  (От противного). Представим $f(X) = G_1 \cup G_2$, где $G_1 = \CG_i \cap f(X),
  G_i \neq \varnothing, i = 1, 2, G_1 \cap G_2 = \varnothing$. Тогда
  $f\rev(\CG_1) \cup f\rev(\CG_2) = X$. Противоречие.
}

\jth[Больцано-Коши в $\R^n$]{
  Если $E \subset \R^n$ связно, $f: E \to \R$ непрерывна, то
  \[ \forall A, B: \exists \vc a, \vc b \in E: f(\vc a) = A \land f(\vc b) = B
  \then\forall \gamma \in (A, B) \hk \exists \vc c \in E: f(\vc c) = \gamma \]
}

\jprf{
  По \hyperref[lemma6.1.3]{лемме 3}, $f(E)$ связно, тогда по
  \hyperref[lemma6.1.2]{лемме 2} $f(E)$ --- промежуток. В таком случае
  требуемое следует из определения промежутка.
}

\jth{
  Каждое открытое связное множество в $\R^n$ линейно связно.
}

\jprf{
  (От противного). Пусть $G$ --- открытое и связное множество. Тогда если $G$ не
  линейно связное, то $\exists \vc x_1, \vc x_2 \in G$ такие, что их нельзя
  соединить кривой в $G$. Введем следующие множества:
  \[  G_1 := \fc{\vc x \in G: \vc x \text{ нельзя соединить с $\vc x_1$ кривой в
  $G$}};\ \ G_2 = G \setminus G_1 \]
  Данные множества очевидно непустые и не пересекаются.

  Докажем, что $G_2$ открыто в $G$. Пусть $\vc x_0 \in G_2 \subset G$ тогда
  $\exists \epsilon > 0: U_\epsilon(\vc x_0) \subset G$. Отсюда $U_\epsilon(\vc
  x_0) \subset G_2$, значит $G_2$ --- открыто. Теперь докажем, что $G_1$
  открыто. От противного,
  \[ \exists \vc x_0 \in G_1: \forall \epsilon > 0
    \hk U_\epsilon(\vc x_0) \cap G_2
  \neq \varnothing \]
  \[ \exists \epsilon_1 > 0: U_{\epsilon_1}(\vc x_0) \subset G
  \hk \vc y \in U_{\epsilon_1}(\vc x_0) \cap G_2\]
  Противоречие.

  \begin{figure}[ht]
    \centering
    \begin{subfigure}{0.4\textwidth}
      \centering
      \tk[draw]{
        \tp[fill=black!10]{(1, .5) circle (1)}
        \tp[fill]{(-1, -1) circle (.03) node[above left] {$\vc x_1$}}
        \tp[fill]{(1, .5) circle (.03) node[below right] {$\vc x_0$}}
        \tp[draw, -Stealth]{(-1, -1) ..controls (-.7, 0) and (.4, -.7) .. (1, .5)}
        \tp[fill]{(.7, 1.2) circle (.03)}
        \tp[draw]{(1, .5)--(.7, 1.2)}
        \tp{(-.6, 0) node {$\gamma \subset G$}}
      }
      \caption{Открытость $G_2$}
    \end{subfigure}
    \begin{subfigure}{0.4\textwidth}
      \centering
      \tk{
        \tp[fill=black!10]{(1, .5) circle (1)}
        \tp[fill]{(-1, -1) circle (.03) node[above left] {$\vc x_1$}}
        \tp[fill]{(1, .5) circle (.03) node[below right] {$\vc x_0$}}
        \tp[fill]{(.7, 1.2) circle (.03) node[above left] {$\vc y_0$}}
        \tp[draw, -Stealth]{(-1, -1) ..controls (.1, -.3) and (-.5, .3) .. (.7, 1.2)}
        \tp{(-.9, .3) node {$\gamma \subset G$}}
        \tp[draw]{(1, .5)--(.7, 1.2)}
      }
      \caption{Открытость $G_1$}
    \end{subfigure}
    \caption{Примеры на $\R^2$ для \hyperref[theorem6.1.6]{теоремы 6}}
  \end{figure}
}


\section{Дифференцируемость функций многих переменных}

\jdf[Дифференцируемость в точке]{
  Пусть $f$ --- действительнозначная функция, определенная в некоторой
  окрестности точки $\vc x_0 \in \R^n$.
  Тогда $f$ называется дифференцируемой в точке $\vc x_0$, если существует
  вектор $\vc A \in \R^n$ такой, что полное приращение функции $f$ в точке
  $\vc x_0$, обозначаемое $\dt y := f(\vc x_0 + \dt \vc x) - f(\vc x_0)$
  представимо в виде $\dt y = (\vc A, \dt \vc x) + o(\fp{\dt \vc x}), \dt \vc x
\to \vc 0$. Кроме того:

  \jit{
    \item $\dt \vc x$ называется вектором приращений аргументов.
    \item $(\vc A, \dt \vc x) = dy = df$ называется дифференциалом функции $f$.
    \item $\vc A$ называется градиентом $f$ и обозначается $\grad f$.
    \item Вектор приращений независимых аргументов считается вектором
      дифференциалов этих переменных, $\dt \vc x = d\vc x$.
  }
}

\jdf[Частиченое приращение функции]{
  Частичным (частным) приращением функции $f$ в точке $\vc x_0 = (x_{1, 0},
  \ldots, x_{n, 0})$ по переменной $x_j$ называется $\dt_j f = f(\vc x_0 +
  \dt_j \vc x) = f(\vc x_0)$, где $\dt_j \vc x = (0, \ldots, 0, \dt x_j, 0,
  \ldots, 0), j = 1, \ldots, n$.

  $\dt_j f$ является приращением функции одного переменного
  \[ \phi_j(x_j) := f(x_{1, 0}, \ldots, x_{j - 1, 0}, x_j,
  x_{j + 1, 0}, \ldots, x_{n, 0}) \]
  то есть $\dt_j f = \dt \phi_j(x_{j, 0})$.
}

\jdf[Частная производная]{
  Частной производной функции $f$ в точке $\vc x_0$ по переменной $x_j$
  называется производная (если она существует) функции $\phi_j$ в точке $x_{j,
  0}$. Обозначается:
  \[ \frac{\pal f}{\pal x_j} (\vc x_0) := \frac{d\phi_j}{dx_j} (x_{j, 0}) \]
}

\jth{
  Если $f$ дифференцируема в точке $\vc x_0$, то она имеет частные производные в
  этой точке по всем переменным $x_j, j = 1, \ldots, n$, причем,
  \[ \grad f(\vc x_0) = \fd{\frac{\pal f}{\pal x_1}(\vc x_0), \ldots,
  \frac{\pal f}{\pal x_1}(\vc x_0)} \]
  Обратное, вообще говоря, не верно.
}

\jprf{
  Пусть $\dt \vc x := \dt_j \vc x$, тогда
  \[ \dt y = f(\vc x_0 + \dt_j \vc x)
  - f(\vc x_0) = \dt_j f = \dt \phi_j (x_{j, 0}) = (A, \dt_j \vc x) + o(\fp{
  \dt_j \vc x}) = A_j\dt x_j + o(\fp{\dt x_j}) \]
  \[ \fp{\dt_j \vc x} = \fp{\dt x_j} \tto 0 \iff \dt x_j \tto 0 \]
  Тогда $\phi_j$ дифференцируема в точке $x_{j, 0}$, причем
  $\phi_j(x_{j, 0}) = A_j = \frac{\pal f}{\pal x_j} (\vc x_0)$.
}

\jexm{
  \[ f(x, y) = \left\{\jarr{c c}{\frac{xy}{x^2 + y^2}, & (x, y) \neq (0, 0) \\
  0, & (x, y) = (0, 0)}\right. \]
  Положим $\vc x_0 = (0, 0)$, тогда
  \[ \frac{\pal f}{\pal x}(0, 0) = \lm[\dt x \to 0] \frac{f(\dt x, 0) - f(0,
  0)}{\dt x} = 0 \]
  Таким образом частные производные существуют. С другой стороны
  \[ \dt y = \dt f = \frac{\dt x \cdot \dt y}{(\dt x)^2 + (\dt y)^2} \]
  Хотим доказать, что
  \[ \dt y \neq o(\fp{(\dt x, \dt y)}),\ \ (\dt x, \dt y) \tto 0 \]
  Возьмем вектор $(\dt x, \dt x) \tto (0, 0)$ при $\dt x \to 0$. Тогда
  \[ \dt f = \frac{\dt x \cdot \dt x}{(\dt x)^2 + (\dt x)^2} = \frac{1}{2}
  \neq o(\dt x) \]
  Что означает, что эта функция не является ни дифференцируемой, ни непрерывной.
}

\jexm{
  \[ g(x, y) = \sqrt{\fp{xy}} \]
  Для данной функции
  \[ \frac{\pal g}{\pal x}(0, 0) = \lm[\dt x \to 0] \frac{g(\dt x 0) - g(0, 0)}
  {\dt x} = 0 \]
  На сей раз возьмем $\dt g = \sqrt{\fp{\dt x \cdot \dt y}}$, тогда введем
  полярные координаты
  \[ \left\{\jald{\dt x = r \cos \phi \\ \dt y = r \sin \phi}\right.
  \ \  \fd{r = \sqrt{(\dt x)^2 + (\dt y)^2} = \fp{(\dt x, \dt y)}}  \]
  После такой замены получим
  \[ 0 \le \dt g = r \sqrt{\fp{\cos \phi \sin \phi}} \le r \]
  То есть функция непрерывна в нуле, но в то же время, $\dt g \neq o(r)$, а
  значит не дифференцируема.
}

\jth{
  Если $f$ дифференцируема в $\vc x_0$, то она непрерывна в $\vc x_0$
}

\jprf{
  Хотим доказать, что $\dt y \to 0$ при $\vc{\dt x} \to \vc 0$. Применим
  неравенство Коши-Буняковского-Шварца:
  \[ \sfp{\sfd{\vc A, \dt \vc x}} \le \sfp{\vc A} \cdot \fp{\dt \vc x} \]
  Отсюда следует требуемое.
}

\jth[Достаточное условие дифференцируемости]{
  Если в некоторой окрестности точки $\vc x_0$ $f$ имеет частные производные по
  всем переменным, непрерывные в $\vc x_0$, то $f$ дифференцируема в $\vc x_0$.
}

\jprf{
  \kmlt{\dt y = \\
  = \big(f(x_{1, 0} + \dt x_1, \ldots, x_{n, 0} + \dt x_n) -
  f(x_{1, 0}, x_{2, 0} + \dt x_2, \ldots, x_{n, 0} + \dt x_n)\big) + \\
  + \big(f(x_{1, 0}, x_{2, 0} + \dt x_2, \ldots, x_{n, 0} + \dt x_n) -
  f(x_{1, 0}, x_{2, 0}, x_{3, 0} + \dt x_3 \ldots, x_{n, 0} + \dt x_n)\big) + \\
  + \ldots + \\
  + \big(f(x_{1, 0}, x_{2, 0}, \ldots, x_{n - 1, 0}, x_{n, 0} + \dt x_n) -
  f(x_{1, 0}, \ldots, x_{n, 0})\big) = \\}
  Теперь к каждой паре слагаемых в скобках применим теорему Лагранжа
  \kml{\\= \frac{\pal f}{\pal x_1} (\xi_1, x_{2, 0} + \dt x_2, \ldots, x_{n, 0}
  + \dt x_n) \cdot \dt x_1 + \\
  + \frac{\pal f}{\pal x_2}(x_{1, 0}, \xi_2, x_{3, 0} + \dt x_3, \ldots, x_{n, 0}
  + \dt x_n) \cdot \dt x_2 + \\
  + \ldots + \\
  + \frac{\pal f}{\pal x_n}(x_{1, 0}, \ldots, x_{n - 1, 0}, \xi_n) \cdot \dt x_n=\\}
  \kmlt{\frac{\pal f}{\pal x_1}(\vc x_0)\dt x_1 + \ldots +
  \frac{\pal f}{\pal x_n}(\vc x_0)\dt x_n
  + \fd{\frac{\pal f}{\pal x_1}(\xi_1, x_{2, 0} + \dt x_2, \ldots, x_{n, 0} +
  \dt x_n) - \frac{\pal f}{\pal x_n}(\vc x_0)} + \ldots + \\
  + \fd{\frac{\pal f}{\pal x_1}(x_{1, 0}, x_{2, 0} + \dt x_2, \ldots, x_{n - 1, 0} +
  \dt x_{n - 1} + \xi_n) - \frac{\pal f}{\pal x_n}(\vc x_0)} \fuck}
}

\tpic{Иллюстрация к \hyperref[theorem6.2.3]{теореме 3}}{
  \tdraw[-Stealth]{(-2.8, -2)--(2.8, -2) node[below left] {$x_n$}}
  \tdraw[-Stealth]{(-2, -2.8)--(-2, 2.2)}
  \tp{(-2, -2) node[below left] {$O$}}
  \tp[fill=black!10]{(0, 0) circle (1.4)}
  \tp[fill]{(0, 0) circle (.03) node[below left] {$\vc x_0$}}
  \tp[fill]{($(0, 0)+(1.1, .6)$) circle (.03) node[above right] {$\vc x_0 + \dt
  \vc x$}}
  \tp[draw]{(.5, -2) node[below] {$\xi_n$}--+(0, .1)--+(0, -.1)}
}
