\lectureinfo{4}{13.02.2026}

\jnt{
  Условие \hyperref[theorem6.2.3]{теоремы 3} не является необходимым.
}

\jexm{
  \[ y = \sqrt[3]{x^2y^2} \]
}

\jnt{
  Также как и для функций одного переменного, из дифференцируемости $f, g$ в
  точке $\vc x_0$ следует дифференцируемость в точке $\vc x_0$
  \jit{
    \item $f \pm g$,
    \item $f \cdot g$,
    \item $\displaystyle \frac{f}{g}$ при $g(\vc x_0) \neq 0$,
  }
  а также следующие равенства в точке $\vc x_0$
  \jit{
    \item $d(f \pm g) = df \pm dg$,
    \item $d(f \cdot g) = g \cdot df + f \cdot dg$,
    \item $\displaystyle d\fd{\frac{f}{g}} = \frac{g \cdot df - f \cdot dg}{g^2}$
      при $g(\vc x_0) \neq 0$.
  }
}

\jth{
  Если $f$ дифференцируема в точке $\vc y_0 \in \R^m$, а функции $g_j(\vc x)$
  дифференцируемы в точке $\vc x_0 \in \R^n$, причем $g_j(\vc x_0) = y_{j, 0}$,
  $\vc y_0 = (y_{1, 0}, \ldots, y_{n, 0}),\; (j = 1, \ldots, m)$, то сложная
  функция $h(\vc x) = f(g_1(\vc x), \ldots, g_m(\vc x))$ дифференцируема в точке
  $\vc x_0$, причем
  \[ \grad h(\vc x_0) = \grad f(\vc y_0) \cdot 
  \fd{\jarr{c}{\grad g_1(\vc x_0) \\ \vdots \\ \grad g_m(\vc x_0)}}\]
}

\jprf{
  \kmlt{\dt h(\vc x_0) = f(g_1(\vc x_0 + \dt \vc x), \ldots, g_m(\vc x_0 + \dt \vc x)) -
  f(g_1(\vc x_0), \ldots, g_m(\vc x_0)) = \\
= \dt f(\vc y_0) = (\grad f(\vc y_0), \dt y) + o(\fp{\dt y}) = \\
  = \grad f(\vc y_0) \cdot
  \fd{\jarr{c}{(\grad g_1(\vc x_0), \dt \vc x) + o_1(\fp{\dt \vc x}) \\ \vdots \\
  (\grad g_m(\vc x_0), \dt \vc x) + o_m(\fp{\dt \vc x})}} +
  o(\fp{\dt \vc y}) = \\
  = \grad f(\vc y_0) \cdot
  \fd{\jarr{c}{(\grad g_1(\vc x_0), \dt \vc x) \\ \vdots \\ \grad g_m(\vc x_0), \dt \vc x)}} +
  \grad f(\vc y_0) \cdot
  \fd{\jarr{c}{o_1(\fp{\dt \vc x}) \\ \vdots \\ o_m(\fp{\dt \vc x})}} +
  o(\fp{\dt y}) = \\
  = (\grad(\vc x_0), \dt \vc x) + o(\fp{\dt \vc x}),\ \ \dt \vc x \to 0}
  Таким образом, теорема доказана.
}

\jcrl{
  (Инвариантность формы первого дифференциала). Формула для дифференциала
  функции $f(\vc x): df = \sm{j = 1}{n} \frac{\pal f}{\pal x_j} dx_j$
  справедлива как в случае, когда $x_1, \ldots, x_n$ --- независимые переменные,
  так и в случае, когда $x_j$ --- дифференцируемые функции от других независимых
  переменных.
}

\jprf{
  Пусть $f(\vc x) = f(g_1(\vc t), \ldots, g_n(\vc t)) =: h(\vc t)$. Тогда
  \kmlt{df =
  dh = \sm{k = 1}{m} \frac{\pal h}{\pal t_k} dt_k = \sm{k = 1}{m} \sm{j = 1}{n}
  \frac{\pal f}{\pal x_j} \cdot \frac{\pal g_j}{\pal t_k} dt_k = \\ = \sm{j = 1}{n}
  \frac{\pal f}{\pal x_j}\fd{\sm{k = 1}{m} \frac{\pal g_j}{\pal t_k} dt_k} =
  \sm{j = 1}{n} \frac{\pal f}{\pal x_j} dg_j = \sm{j = 1}{n} \frac{\pal f}{\pal
  x_j} dx_j}
}

\section{Геометрический смысл градиента и дифференцируемости}

\jdf[Область]{
  Область --- это открытое связное множество.
}

\jdf[Замкнутая область]{
  Замкнутая область --- это замыкание области.
}

\jdf[Параметрически заданная поверхность]{
  Параметрически заданной поверхностью называется множество точек пространства
  $\R^3$, заданных с помощью непрерывных функций $x = x(u, v),\; y = y(u, v),\;
  z = z(u, v)$ на замкнутой области $\ol D \subset \R^2$.
}

\jdf[Касательная плоскость]{
  Плоскость, проходящая через точку $N_0$ параметрически заданной поверхности
  называется касательной к этой поверхности, если угол между ней и любой
  секущей, проходящей через точки $N_0$ и $N$ поверхности стремится к нулю при
  $N \to N_0$.
}

\jdf{
  График функции $z = f(x, y)$, где $(x, y) \in \ol \Omega$:
  \[ \fc{(x, y, z): (x, y) \in \ol \Omega, z = f(x, y)} \]
}

\jth{
  Если $f(x, y)$ дифференцируема в точке $(x_0, y_0)$, то ее график имеет
  касательную плоскость в точке $(x_0, y_0, z_0),\; z_0 = f(x_0, y_0)$,
  заданную уравнением
  \[ z - z_0 = \frac{\pal f}{\pal x}(x_0, y_0)(x - x_0) +
  \frac{\pal f}{\pal y}(x_0, y_0)(y - y_0) \]
  Ясно, что заданная плоскость проходит через точку $N_0 = (x_0, y_0, z_0)$.
  Нормаль к ней имеет координаты $\fd{\frac{\pal f}{\pal x}(x_0, y_0),
  \frac{\pal f}{\pal y}(x_0, y_0), -1} =: \vc n$
}

\jprf{
  \[ \vc{N_0N}: (x - x_0, y - y_0, f(x, y) - f(x_0, y_0)) \]
  Надо доказать, что
  \[ \frac{(\vc n, \vc{N_0N})}{\fp{\vc n} \cdot \fp{\vc{N_0N}}} \to \vc 0 \]
  при $(x - x_0, y - y_0) \to 0$.
  Тогда
  \kmlt{\fp{\frac{(\vc n, \vc{N_0N})}{\fp{\vc n} \cdot \fp{\vc{N_0N}}} =
  \frac{\frac{\pal f}{\pal x}(x_0, y_0)(x - x_0) +
  \frac{\pal f}{\pal y}(x_0, y_0)(x - x_0) - (f(x, y) - f(x_0, y_0))}{
  \fp{\vc n} \cdot \sqrt{(x -x_0)^2 + (y - y_0)^2 + (f(x, y) - f(x_0, y_0))^2}
  }} \le \\
  \le \frac{\fp{o(\fp{\dt x})}}{1 \cdot \fp{\dt x}}}
}

\jdf{
  Производной функции $f$ в точке $\vc x_0 \in \R^n$ по направлению $\vc l \in
  \R^n \setminus \fc{\vc 0}$ называется предел (если он существует)
  \[ \frac{\pal f}{\pal \vc l}(\vc x_0) =
  \lm[t \to +0] \frac{f(\vc x_0 + t\vc l) - f(\vc x_0)}{t}\]
}

\jth{
  Если $f$ дифференцируема в точке $\vc x_0 \in \R^n$, то она имеет производные
  в этой точке по всем направлениям $\vc l \in \R^n \setminus \fc{\vc 0}$,
  причем $\frac{\pal f}{\pal x} (\vc x_0) = (\grad f(\vc x_0), \vc l)$
}

\jprf{
  \[ \frac{f(\vc x_0 + t \vc l) - f(\vc x_0)}{t} =
  \frac{(\grad f(\vc x_0), t \vc l) + o(|t \vc l|)}{t}
  = (\grad f(\vc x_0), \vc l) + o(1) \]
}
