\documentclass[12pt, a4paper]{extreport}

%%% Предметный указатель
\usepackage{imakeidx}
\makeindex[columns=2, intoc, options={-s preamble/style.ist}]

%%% Приложения
\usepackage[toc, title]{appendix}
\renewcommand{\appendixtocname}{Приложения}
\renewcommand{\appendixpagename}{Приложения}

%%% Русский язык и шрифты
\usepackage{fontspec}
\usepackage{indentfirst}
\usepackage[english,main=russian]{babel}

\defaultfontfeatures{Rederer=Basic,Ligatures={TeX}}
\setmainfont{CMU Serif}
\setsansfont{CMU Sans Serif}
\setmonofont{CMU Typewriter Text}

%%% Математика
\usepackage{amsmath,amsfonts,amssymb,amsthm,mathtools}
\usepackage{icomma}
\usepackage{stmaryrd}

%%% Рисование графики
\usepackage{tikz}        % Графический пакет tikz
\usepackage{tikz-cd}     % Коммутативные диаграммы
\usepackage{tkz-euclide} % Геометрия
\usepackage{stackengine} % Многострочные тексты в картинках
\usetikzlibrary{angles, babel, quotes, matrix, calc, shapes, shapes.symbols,
positioning, fit}

%%% Перенос знаков в формулах (по Львовскому)
\newcommand{\hm}[1]{#1\nobreak\discretionary{}{\hbox{$\mathsurround=0pt #1$}}{}}

%%% Работа с картинками
\usepackage{graphicx}    % Для вставки рисунков
\setlength\fboxsep{3pt}  % Отступ рамки \fbox{} от рисунка
\setlength\fboxrule{1pt} % Толщина линий рамки \fbox{}
\usepackage{wrapfig}     % Обтекание рисунков текстом

%%% Работа с таблицами
\usepackage{array,tabularx,tabulary,booktabs, arydshln} % Дополнительная работа с таблицами
\usepackage{longtable}                                  % Длинные таблицы
\usepackage{multirow}                                   % Слияние строк в таблице

%%% Оформление страницы
\usepackage{geometry}     % Простой способ задавать поля
\usepackage{setspace}     % Интерлиньяж
\usepackage{enumitem}     % Настройка окружений itemize и enumerate
\setlist{leftmargin=25pt} % Отступы в itemize и enumerate

\geometry{top=25mm}    % Поля сверху страницы
\geometry{bottom=30mm} % Поля снизу страницы
\geometry{left=20mm}   % Поля слева страницы
\geometry{right=20mm}  % Поля справа страницы

\setlength\parindent{15pt}        % Устанавливает длину красной строки 15pt
\linespread{1.3}                  % Коэффициент межстрочного интервала
\usepackage{multicol}             % Для текста в нескольких колонках
\usepackage{soul}                 % Модификаторы начертания

%%% Оглавление
\usepackage{tocloft}
\tocloftpagestyle{main}

%%%% Нумерация уравнений
%\makeatletter
%\def\eqref{\@ifstar\@eqref\@@eqref}
%\def\@eqref#1{\textup{\tagform@{\ref*{#1}}}}
%\def\@@eqref#1{\textup{\tagform@{\ref{#1}}}}
%\makeatother                      % \eqref* без гиперссылки
%\numberwithin{equation}{section}  % Нумерация вида (номер_секции).(номер_уравнения)
%\mathtoolsset{showonlyrefs=false} % Номера только у формул с \eqref{} в тексте.

%%% Гиперссылки
\usepackage{hyperref}
\usepackage[dvipsnames,svgnames,table,rgb]{xcolor}
\hypersetup{
	unicode=true,            % русские буквы в разделах PDF
	colorlinks=true,         % Цветные ссылки вместо ссылок в рамках
	linkcolor=black!15!blue, % Внутренние ссылки
	citecolor=green,         % Ссылки на библиографию
	filecolor=magenta,       % Ссылки на файлы
	urlcolor=NavyBlue,       % Ссылки на URL
}

%%% Теоремы
\theoremstyle{plain}
\newtheorem  {theorem}     {Теорема}    [section]
\newtheorem* {theorem*}    {Теорема}
\newtheorem  {lemma}       {Лемма}      [section]
\newtheorem* {lemma*}      {Лемма}
\newtheorem  {proposition} {Утверждение}[section]
\newtheorem* {proposition*}{Утверждение}
\newtheorem* {corollary}   {Следствие}

\theoremstyle{definition}
\newtheorem  {definition}  {Определение}[section]
\newtheorem* {definition*} {Определение}
\newtheorem  {exercise}    {Упражнение}
\newtheorem* {exercise*}   {Упражнение}
\newtheorem  {problem}     {Задача}
\newtheorem* {problem*}    {Задача}
\newtheorem  {reminder}    {Напоминание}
\newtheorem* {reminder*}   {Напоминание}
\newtheorem  {example}     {Пример}
\newtheorem* {example*}    {Пример}
\newtheorem* {examples}    {Примеры}

\theoremstyle{remark}
\newtheorem  {note}        {Замечание}
\newtheorem* {note*}       {Замечание}
\newtheorem* {solution}    {Решение}

%%% Свои символы и команды
\renewcommand{\epsilon} {\varepsilon}
\renewcommand{\phi}     {\varphi}
\renewcommand{\kappa}   {\varkappa}
\renewcommand{\le}      {\leqslant}
\renewcommand{\leq}     {\leqslant}
\renewcommand{\ge}      {\geqslant}
\renewcommand{\geq}     {\geqslant}
\renewcommand{\emptyset}{\varnothing}

\DeclareMathOperator{\grad} {grad}
\DeclareMathOperator{\ord}  {ord}
\DeclareMathOperator{\const}{const}
\DeclareMathOperator{\Char} {char}
\DeclareMathOperator{\chr}  {char}
\DeclareMathOperator{\cl}   {cl}
\DeclareMathOperator{\rk}   {rk}
\DeclareMathOperator{\pr}   {pr}
\DeclareMathOperator{\Ker}  {Ker}
\DeclareMathOperator{\Int}  {int}
\DeclareMathOperator{\sign} {sign}
\DeclareMathOperator{\sgn}  {sgn}
\DeclareMathOperator{\im}   {Im}
\DeclareMathOperator{\re}   {Re}
\DeclareMathOperator{\id}   {id}

\newcommand{\N} {\mathbb{N}}
\newcommand{\Z} {\mathbb{Z}}
\newcommand{\Q} {\mathbb{Q}}
\newcommand{\R} {\mathbb{R}}
\newcommand{\Cm}{\mathbb{C}}
\newcommand{\F} {\mathbb{F}}

\newcommand{\CF}{\mathcal{F}}
\newcommand{\CL}{\mathcal{L}}
\newcommand{\CT}{\mathcal{T}}
\newcommand{\CG}{\mathcal{G}}

\newcommand{\FI}{\mathfrak{I}}

\newcommand{\eps}  {\epsilon}

\newcommand{\hk}   {\;\hookrightarrow\;}
\newcommand{\dt}   {\Delta}
\newcommand{\pal}  {\partial}

\newcommand{\undersetifgiven}[2]{
  \ifstrempty{#1}{#2}{\ensuremath{\underset{#1}{#2}}}
}

\renewcommand{\to} [1][]{\undersetifgiven{#1}{\rightarrow}}
\newcommand{\from} [1][]{\undersetifgiven{#1}{\leftarrow}}
\newcommand{\tto}  [1][]{\undersetifgiven{#1}{\longrightarrow}}
\newcommand{\ffrom}[1][]{\undersetifgiven{#1}{\longleftarrow}}
\newcommand{\is}   [1][]{\undersetifgiven{#1}{\leftrightarrow}}
\newcommand{\iss}  [1][]{\undersetifgiven{#1}{\longleftrightarrow}}
\newcommand{\To}   [1][]{\undersetifgiven{#1}{\Longrightarrow}}
\newcommand{\From} [1][]{\undersetifgiven{#1}{\Longleftarrow}}

\renewcommand{\implies}{\;\To\;}

\newcommand{\rev}      {^{-1}}
\newcommand{\ol}  [1]  {\overline{#1}}
\newcommand{\unl} [1]  {\underline{#1}}
\newcommand{\dv}       {\mid}
\newcommand{\ndv}      {\nmid}
\newcommand{\supr}[1]  {^{\fd{#1}}}
\newcommand{\then}     {\;\,}
\newcommand{\lm}  [1][]{\displaystyle \lim_{#1}}
\newcommand{\dlm} [2]  {\displaystyle \lim_{\substack{#1 \\ #2}}}
\newcommand{\sm}  [2]  {\displaystyle \sum_{#1}^{#2}}
\newcommand{\smm} [2]  {\sum\limits_{#1}^{#2}}
\newcommand{\prd} [2]  {\displaystyle \prod_{#1}^{#2}}
\newcommand{\prdm}[2]  {\prod\limits_{#1}^{#2}}

\newcommand{\theoremindex}[6]{
  \ifstrempty{#1} {
    \index{#4#2!
      \arabic{chapter}?\arabic{section}!
      \alph{#6}@#3 #5}
  }{
    \index{#4#2!
      \arabic{chapter}?\arabic{section}!
      \alph{#6}@#3 #5. #1}
  }
}

\newcommand{\mylbl}[1]{\label{#1\arabic{chapter}.\arabic{section}.\arabic{#1}}}

%%% Окружения (начинаются с j или k)
\newcommand{\jtheorem}[2][]{
  \begin{theorem}[#1]
    #2 \theoremindex{#1}{Теоремы}{Теорема}{.@}{\thetheorem}{theorem}
    \mylbl{theorem}
  \end{theorem}}
\newcommand{\jtheor}      [2][]{\jtheorem[#1]{#2}}
\newcommand{\jthrm}       [2][]{\jtheorem[#1]{#2}}
\newcommand{\jth}         [2][]{\jtheorem[#1]{#2}}

\newcommand{\ktheorem}    [2][]{\begin{theorem*}[#1]#2\end{theorem*}}
\newcommand{\ktheor}      [2][]{\ktheorem[#1]{#2}}
\newcommand{\kthrm}       [2][]{\ktheorem[#1]{#2}}
\newcommand{\kth}         [2][]{\ktheorem[#1]{#2}}

\newcommand{\jlemma}[2][]{
  \begin{lemma}[#1]
    #2 \theoremindex{#1}{Леммы}{Лемма}{}{\thelemma}{lemma}
    \mylbl{lemma}
  \end{lemma} }
\newcommand{\jlmm}        [2][]{\jlemma[#1]{#2}}
\newcommand{\jlm}         [2][]{\jlemma[#1]{#2}}

\newcommand{\klemma}      [2][]{\begin{lemma*}[#1]#2\end{lemma*}}
\newcommand{\klmm}        [2][]{\klemma[#1]{#2}}
\newcommand{\klm}         [2][]{\klemma[#1]{#2}}

\newcommand{\jproposition}[2][]{
  \begin{proposition}[#1]
    #2 \theoremindex{#1}{Утверждения}{Утверждение}{}{\theproposition}{proposition}
    \mylbl{proposition}
  \end{proposition}}
\newcommand{\jpropos}     [2][]{\jproposition[#1]{#2}}
\newcommand{\jprop}       [2][]{\jproposition[#1]{#2}}
\newcommand{\jprp}        [2][]{\jproposition[#1]{#2}}

\newcommand{\kproposition}[2][]{\begin{proposition*}[#1]#2\end{proposition*}}
\newcommand{\kpropos}     [2][]{\kproposition[#1]{#2}}
\newcommand{\kprop}       [2][]{\kproposition[#1]{#2}}
\newcommand{\kprp}        [2][]{\kproposition[#1]{#2}}

\newcommand{\jexercise}   [1]{\begin{exercise}#1\mylbl{exercise}\end{exercise}}
\newcommand{\jexrc}       [1]{\jexercise{#1}}
\newcommand{\jexc}        [1]{\jexercise{#1}}
\newcommand{\jexr}        [1]{\jexercise{#1}}

\newcommand{\kexercise}   [1]{\begin{exercise*}#1\end{exercise*}}
\newcommand{\kexrc}       [1]{\kexercise{#1}}
\newcommand{\kexc}        [1]{\kexercise{#1}}
\newcommand{\kexr}        [1]{\kexercise{#1}}

\newcommand{\jproblem}    [1]{\begin{problem}#1\mylbl{problem}\end{problem}}
\newcommand{\jprb}        [1]{\jproblem{#1}}

\newcommand{\kproblem}    [1]{\begin{problem}#1\end{problem}}
\newcommand{\kprb}        [1]{\kproblem{#1}}

\newcommand{\jproof}      [1]{\begin{proof}#1\end{proof}}
\newcommand{\jprof}       [1]{\jproof{#1}}
\newcommand{\jprf}        [1]{\jproof{#1}}

\newcommand{\jdefinition} [2][]{
  \begin{definition}[#1]
    #2 \theoremindex{#1}{Определения}{Определение}{}{\thedefinition}{definition}
    \mylbl{definition}
  \end{definition}}
\newcommand{\jdefine}     [2][]{\jdefinition[#1]{#2}}
\newcommand{\jdf}         [2][]{\jdefinition[#1]{#2}}

\newcommand{\kdefinition} [2][]{\begin{definition*}[#1]#2\end{definition*}}
\newcommand{\kdefine}     [2][]{\kdefinition[#1]{#2}}
\newcommand{\kdf}         [2][]{\kdefinition[#1]{#2}}

\newcommand{\jcorollary}  [1]{\begin{corollary}#1\end{corollary}}
\newcommand{\jcrlr}       [1]{\jcorollary{#1}}
\newcommand{\jcrl}        [1]{\jcorollary{#1}}
\newcommand{\jcr}         [1]{\jcorollary{#1}}

\newcommand{\jnote}       [1]{\begin{note}#1\mylbl{note}\end{note}}
\newcommand{\jnt}         [1]{\jnote{#1}}

\newcommand{\knote}       [1]{\begin{note*}#1\end{note*}}
\newcommand{\knt}         [1]{\knote{#1}}

\newcommand{\jreminder}   [1]{\begin{reminder}#1\mylbl{reminder}\end{reminder}}
\newcommand{\jremind}     [1]{\jreminder{#1}}
\newcommand{\jrmnd}       [1]{\jreminder{#1}}
\newcommand{\jrmd}        [1]{\jreminder{#1}}

\newcommand{\kreminder}   [1]{\begin{reminder*}#1\end{reminder*}}
\newcommand{\kremind}     [1]{\kreminder{#1}}
\newcommand{\krmnd}       [1]{\kreminder{#1}}
\newcommand{\krmd}        [1]{\kreminder{#1}}

\newcommand{\jexample}    [1]{\begin{example}#1\mylbl{example}\end{example}}
\newcommand{\jexmpl}      [1]{\jexample{#1}}
\newcommand{\jexmp}       [1]{\jexample{#1}}
\newcommand{\jexm}        [1]{\jexample{#1}}

\newcommand{\kexample}    [1]{\begin{example*}#1\end{example*}}
\newcommand{\kexmpl}      [1]{\kexample{#1}}
\newcommand{\kexmp}       [1]{\kexample{#1}}
\newcommand{\kexm}        [1]{\kexample{#1}}

\newcommand{\jexamples}   [1]{\begin{examples}#1\end{examples}}
\newcommand{\jexmpls}     [1]{\jexamples{#1}}
\newcommand{\jexmps}      [1]{\jexamples{#1}}
\newcommand{\jexms}       [1]{\jexamples{#1}}

\newcommand{\jsolution}   [1]{\begin{solution}#1\end{solution}}
\newcommand{\jsolve}      [1]{\jsolution{#1}}
\newcommand{\jsol}        [1]{\jsolution{#1}}
\newcommand{\jslv}        [1]{\jsolution{#1}}

\newcommand{\jalign}      [1]{\begin{align}#1\mylbl{equation}\end{align}}
\newcommand{\jalgn}       [1]{\jalign{#1}}
\newcommand{\jalig}       [1]{\jalign{#1}}
\newcommand{\jalg}        [1]{\jalign{#1}}

\newcommand{\kalign}      [1]{\begin{align*}#1\end{align*}}
\newcommand{\kalgn}       [1]{\kalign{#1}}
\newcommand{\kalig}       [1]{\kalign{#1}}
\newcommand{\kalg}        [1]{\kalign{#1}}

\newcommand{\jaligned}    [1]{\begin{aligned}#1\end{aligned}}
\newcommand{\jalgnd}      [1]{\jaligned{#1}}
\newcommand{\jaligd}      [1]{\jaligned{#1}}
\newcommand{\jalgd}       [1]{\jaligned{#1}}
\newcommand{\jald}        [1]{\jaligned{#1}}

\newcommand{\jenumerate}  [1]{\begin{enumerate}#1\end{enumerate}}
\newcommand{\jenum}       [1]{\jenumerate{#1}}
\newcommand{\jenm}        [1]{\jenumerate{#1}}
\newcommand{\jen}         [1]{\jenumerate{#1}}

\newcommand{\jitemize}    [1]{\begin{itemize}#1\end{itemize}}
\newcommand{\jitems}      [1]{\jitemize{#1}}
\newcommand{\jitem}       [1]{\jitemize{#1}}
\newcommand{\jitm}        [1]{\jitemize{#1}}
\newcommand{\jit}         [1]{\jitemize{#1}}

\newcommand{\jequation}   [1]{\begin{equation}#1\mylbl{equation}\end{equation}}
\newcommand{\jeq}         [1]{\jequation{#1}}

\newcommand{\kequation}   [1]{\begin{equation*}#1\end{equation*}}
\newcommand{\keq}         [1]{\kequation{#1}}

\newcommand{\jarray}      [2]{\begin{array}{#1}#2\end{array}}
\newcommand{\jarr}        [2]{\jarray{#1}{#2}}

\newcommand{\jmultline}   [1]{\begin{multline}#1\mylbl{equation}\end{multline}}
\newcommand{\jml}         [1]{\jmultline{#1}}
\newcommand{\jmltln}      [1]{\jmultline{#1}}
\newcommand{\jmlt}        [1]{\jmultline{#1}}
\newcommand{\jmline}      [1]{\jmultline{#1}}
\newcommand{\jmln}        [1]{\jmultline{#1}}

\newcommand{\kmultline}   [1]{\begin{multline*}#1\end{multline*}}
\newcommand{\kml}         [1]{\kmultline{#1}}
\newcommand{\kmltln}      [1]{\kmultline{#1}}
\newcommand{\kmlt}        [1]{\kmultline{#1}}
\newcommand{\kmline}      [1]{\kmultline{#1}}
\newcommand{\kmln}        [1]{\kmultline{#1}}

\newcommand{\jcustom}     [2][]{\begin{#1}#2\end{#1}}
\newcommand{\jcustm}      [2][]{\jcustom[#1]{#2}}
\newcommand{\jcust}       [2][]{\jcustom[#1]{#2}}
\newcommand{\jcstm}       [2][]{\jcustom[#1]{#2}}
\newcommand{\jcst}        [2][]{\jcustom[#1]{#2}}
\newcommand{\jcs}         [2][]{\jcustom[#1]{#2}}
\newcommand{\jct}         [2][]{\jcustom[#1]{#2}}

%%% Скобки (начинаются с f)
\newcommand{\fd} [1]{\left( #1 \right)}   % Parentheses
\newcommand{\fb} [1]{\left[ #1 \right]}   % Brackets
\newcommand{\fp} [1]{\left| #1 \right|}   % Pipes
\newcommand{\fpp}[1]{\left\| #1 \right\|} % double PiPes
\newcommand{\fdp}[1]{\left\| #1 \right\|} % Double Pipes
\newcommand{\fP} [1]{\left\| #1 \right\|} % double Pipes
\newcommand{\fc} [1]{\left\{ #1 \right\}} % Curly braces
\newcommand{\fa} [1]{\langle #1 \rangle}  % Angles

%%% Tikz (начинаются с t) WIP!!!!
\newcommand{\tk}  [2][]{
  \begin{tikzpicture}[remember picture,#1]
    #2
  \end{tikzpicture}
}

\newcommand{\tpic}[3][]{
  \begin{figure}[ht]
    \centering
    \tk[#1]{#3}
    \caption{#2}
    \mylbl{figure}
  \end{figure}
}

\newcommand{\dpic}[2][]{
  \begin{figure}[ht]
    \centering
    \tk[#1]{#2}
    \mylbl{figure}
  \end{figure}
}

\newcommand{\twrap} [5][]{
  \begin{wrapfigure}{#3}{#4}
    \centering
    \tk[#1]{#5}
    \caption{#2}
    \mylbl{figure}
  \end{wrapfigure}\
}

\newcommand{\dwrap} [4][]{
  \begin{wrapfigure}{#2}{#3}
    \centering
    \tk[#1]{#4}
  \end{wrapfigure}\
}

\newcommand{\tpath}  [2][]{\path[#1] #2;}
\newcommand{\tp}     [2][]{\tpath[#1]{#2}}
\newcommand{\tdraw}  [2][]{\draw[#1] #2;}
\newcommand{\tfill}  [2][]{\fill[#1] #2;}
\newcommand{\tnode}  [2][]{\node[#1] {#2};}
\newcommand{\tfor}   [4][]{\foreach[#1] #2 in {#3}{#4}}
\newcommand{\tf}     [4][]{\tfor[#1]{#2}{#3}{#4}}
\newcommand{\tmatrix}[2][]{
  \matrix[ampersand replacement=\&, #1]{\replace{#2}{&}{\&}};
}
\newcommand{\tm}     [2][]{\tmatrix[#1]{#2}}

\newcommand{\vc}[1]{
  % \mathstyle % To debug styles
  {}           % To fix some strange issues
  \ifnum \mathstyle>6
    \tk[baseline]{
      \def\vup{0.16ex}
      \node[inner sep=0pt, anchor=base] (char) {$\scriptscriptstyle #1$};
      \draw[-{Stealth[length=.6mm]}, very thin]
      ($(char.north west) + (0.03,\vup)$) -- ($(char.north east)+(0,\vup)$);
    }
  \else
    \ifnum \mathstyle>3
      \tk[baseline]{
        \def\vup{0.2ex}
        \node[inner sep=0pt, anchor=base] (char) {$\scriptstyle #1$};
        \draw[-{Stealth[length=.8mm]}]
        ($(char.north west) + (0.03,\vup)$) -- ($(char.north east)+(0,\vup)$);
      }
    \else
      \tk[baseline]{
        \def\vup{0.35ex}
        \node[inner sep=0pt, anchor=base] (char) {$#1$};
        \draw[-{Stealth[length=1mm]}]
        ($(char.north west) + (0.03,\vup)$) -- ($(char.north east)+(0,\vup)$);
      }
    \fi
  \fi
}

%%% Replace
\ExplSyntaxOn
\NewDocumentCommand{\replace}{mmm}
 {
  \marian_replace:nnn {#1} {#2} {#3}
 }

\tl_new:N \l_marian_input_text_tl

\cs_new_protected:Npn \marian_replace:nnn #1 #2 #3
 {
  \tl_set:Nn \l_marian_input_text_tl { #1 }
  \tl_replace_all:Nnn \l_marian_input_text_tl { #2 } { #3 }
  \tl_use:N \l_marian_input_text_tl
 }
\ExplSyntaxOff

%%% Метки
\newcommand{\fuck}{ % Ставить в любой непонятной ситуации (для отладки)
  \tk{
    \fill[color=red] (-1mm, 1mm) circle (.1cm);
  }
  \tk[overlay]{
    \coordinate (lm) at ($(current page.west|-0,0)+(10mm - 1mm, 1mm)$);
    \tf{\i}{0,0.1,...,1}{
      \fill[color=red] ($(lm)+(0, \i / 2)$) circle (-4 * 0.02 * \i * \i + 4 * 0.02 * \i + 0.05);
    }
    \fill[color=red] ($(lm)+(0, -2mm)$) circle (.06cm);
  }
}

\usepackage{xstring}

%%% Комманды для предметного указателя
\newcounter{tmpchap}
\newcounter{tmpsec}

\newcommand{\inditn} [1]{\ \hfill
  \setcounter{tmpchap}{0}
  \setcounter{tmpsec}{0}
  {\large \textbf{#1}}
}
\newcommand{\indito} [1]{\ \hfill \vspace{1ex}
  \IfSubStrBefore{#1}{\thetmpchap}{?}{
    \tk[overlay, gray]{\tdraw{(-2.7, 0)--(2.7, 0)}
      \tnode[fill=white]{
        {\footnotesize \S \StrBehind{#1}{?}[\tmpc]\tmpc\setcounter{tmpsec}{\tmpc}}
      }
    }
  }
  {
    \tk[overlay, gray]{\tdraw{(-4, 0)--(4, 0)}
      \tnode[fill=white]{
        {\small Глава \StrBefore{#1}{?}[\tmpp]\tmpp\setcounter{tmpchap}{\tmpp}}
      }
    }
  }
}
\newcommand{\inditt} [1]{\small #1}
\newcommand{\inditx} [1]{#1}

%%% Свои значки в \itemize
\setlist[itemize,1]{label=$*$}
\setlist[itemize,2]{label=$-$}
\setlist[itemize,3]{label=$\centerdot$}
\setlist[itemize,4]{label=$\cdot$}

\setlist[enumerate,1]{label=$\textnormal{\arabic{enumi}.}$}
\setlist[enumerate,2]{label=$\textnormal{\theenumi\arabic{enumii}.}$}
\setlist[enumerate,3]{label=$\textnormal{\theenumii\arabic{enumiii}.}$}

%%% Пометки о начале лекций
\newcommand\lectureinfo[2]{
  \tk[overlay]{
    \tp{(current page.east |- 0,0)+(-20mm, 0)
    node[anchor=east, color=black!50] {\footnotesize{Лекция #1 (#2)}}}
  }
}

%%% Колонтитулы
\newcommand{\ifzr}[3]{\ifnum #1=0 #2 \else #3 \fi}

\usepackage{titleps}
\newpagestyle{main}{
	\setheadrule{0.4pt}
  \sethead{\hyperlink{intro}{$\uparrow$}}{\ifzr{\value{section}}{}{\thesection}\sectiontitle}{\thepage}
}
\pagestyle{main}
