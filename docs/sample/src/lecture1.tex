\chapter{Название главы}

\section{Первый раздел}

\subsection{Подраздел}

\lectureinfo{99}{30.02.2030}

\begin{definition}
  \textit{Определением} называется окружение, в котором определяются новые объекты.
\end{definition}

\section{Второй раздел}

\begin{theorem}[Великая теорема Ферма]
  Уравнение $x^n + y^n = z^n$ не имеет нетривиальных целочисленных решений при $n \in \N$, $n > 2$.
\end{theorem}

\begin{proof}
  Тривиально.
\end{proof}

\jth[Теорема с ссылкой в указателе]{
  У этой теоремы есть ссылка в предметном указателе.
}

\jprf{
  Нетрудно показать, промотав в конец конспекта.
}

\jprp[Простое утверждение]{
  $2 \cdot 2 = 4$
}

\jdf[Матрица]{
  Матрица --- математический объект, записываемый в виде прямоугольной таблицы,
  который представляет собой совокупность строк и столбцов, на пересечении
  которых находятся его элементы.
}

\jlm[Пустая лемма]{}

\jeq{a^2 + b^2 = c^2}

\keq{
  a^n + b^n = c^n
}

\subsection{Примеры векторов}

\jen{
  \item \[ \vc u \]
  \item \[ \vc u_0 + \vc u^2 + \vc{u^2} \]
  \item \[ \vc u' \cdot \vc u''' \]
 \item \[ \vc AB \oplus \vc{AB} \]
}
